\documentclass{article}

\usepackage{fontspec}
\usepackage{fullpage}
\usepackage{multicol}
\usepackage{multirow}
\usepackage{tikz}

\begin{document}

\newfontfamily\swfill{SuttonSignWritingFill.ttf}
\newfontfamily\swline{SuttonSignWritingLine.ttf}
\newcommand{\bul}{\hfil$\bullet$&}
\renewenvironment{glossary}{\begin{multicols}{5}\begin{center}}{\end{center}\end{multicols}}
\setcounter{secnumdepth}{0}
\setlength{\columnseprule}{1pt}

\section{Supplement For Lesson 14}

\subsection{Objectives}

\begin{tabular}{p{1cm}p{14cm}}
\bul I have completed the objectives for this lesson.\\
\bul I know which base symbols are in Symbol Group trunk.\\
\bul I know which base symbols are in Symbol Group limbs.\\
\bul I am able to write and sign the Index Thumb Circle handshape.\\
\bul I am able to write and sign the Index Thumb Cup handshape.\\
\bul I am able to write and sign the Index Thumb Cup Open handshape.\\
\bul I am able to write and sign the Index Thumb Hinge Large handshape.\\
\bul I am able to write and sign the Index Thumb Hinge handshape.\\
\bul I am able to write and sign the Index Thumb Angle handshape.\\
\bul I am able to recognize the vocabulary for this lesson.\\
\bul I am able to read the practice sentences for this lesson.\\
\bul I am able to read the practice story for this lesson.\\
\end{tabular}

\subsection{Symbol Group Trunk}

The twenty-seventh Symbol Group we informally call trunk.
Its official name is ``Brow Eyes Eyegaze'', but you can now list off the symbol groups as ``one, two, three, four, five, six, seven, eight, nine, thumb, contact, fingers, wall, diagonal, floor, curve wall, hit wall, hit floor, curve floor, circles, timing, head, eyes, middle, lips, mouth, trunk''.

\begin{center}
\begin{tabular}{rcrc}
\textbf{Base Symbol}&\textbf{Example}&\textbf{Base Symbol}&\textbf{Example}\\
Shoulder Hip Spine                &B521x502S36d00479x498&Shoulder Hip Positions       &B523x506S36e00477x494\\
Shoulder Hip Move Wall Plane      &B527x505S36f00473x495&Shoulder Hip Move Floor Plane&B527x505S37000474x495\\
Shoulder Tilts (from Waist)       &B525x506S37100475x495&Torso Straight Stretch Wall  &B521x510S37200479x490\\
Torso Curved Bend Wall            &B521x507S37300479x493&Torso Twist Floor Plane      &B521x506S37400479x494\\
Upper Body Tilts (from Hip Joints)&B518x524S37500482x476\\
\end{tabular}
\end{center}

\subsection{Symbol Group Limbs}

The twenty-eighth Symbol Group we informally call limbs.
Its official name is also ``Limbs'', but you can now list off the symbol groups as ``one, two, three, four, five, six, seven, eight, nine, thumb, contact, fingers, wall, diagonal, floor, curve wall, hit wall, hit floor, curve floor, circles, timing, head, eyes, middle, lips, mouth, trunk, limbs''.

\begin{center}
\begin{tabular}{rcrc}
\textbf{Base Symbol}&\textbf{Example}&\textbf{Base Symbol}&\textbf{Example}\\
Limb Combinations&B512x512S37600488x489&Limb Length 1&B501x513S37700499x488\\
Limb Length 2    &B501x516S37800499x484&Limb Length 3&B501x520S37900499x480\\
Limb Length 4    &B501x524S37a00499x476&Limb Length 5&B501x508S37b00499x492\\
Limb Length 6    &B501x506S37c00499x494&Limb Length 7&B501x504S37d00499x496\\
Fingers          &B505x507S37e00496x493\\
\end{tabular}
\end{center}
\begin{tabular}{p{1cm}p{14cm}}
\bul I know which base symbols are in Symbol Group limbs.\\
\bul I am able to write and sign the Index Thumb Circle handshape.\\
\bul I am able to write and sign the Index Thumb Cup handshape.\\
\bul I am able to write and sign the Index Thumb Cup Open handshape.\\
\bul I am able to write and sign the Index Thumb Hinge Large handshape.\\
\bul I am able to write and sign the Index Thumb Hinge handshape.\\
\bul I am able to write and sign the Index Thumb Angle handshape.\\
\bul I am able to recognize the vocabulary for this lesson.\\
\bul I am able to read the practice sentences for this lesson.\\
\bul I am able to read the practice story for this lesson.\\
\end{tabular}
\begin{center}\textbf{\Huge This is where we are!}\end{center}\end{document}
