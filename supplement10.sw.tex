\documentclass{article}

\usepackage{fontspec}
\usepackage{fullpage}
\usepackage{multicol}
\usepackage{multirow}
\usepackage{tikz}

\begin{document}

\newfontfamily\swfill{SuttonSignWritingFill.ttf}
\newfontfamily\swline{SuttonSignWritingLine.ttf}
\newcommand{\bul}{\hfil$\bullet$&}
\renewenvironment{glossary}{\begin{multicols}{5}\begin{center}}{\end{center}\end{multicols}}
\setcounter{secnumdepth}{0}
\setlength{\columnseprule}{1pt}

\section{Supplement For Lesson 10}

\subsection{Objectives}

\begin{tabular}{p{1cm}p{14cm}}
\bul I have completed the objectives for this lesson.\\
\bul I know which base symbols are in Symbol Group curve floor.\\
\bul I know which base symbols are in Symbol Group circles.\\
\bul I am able to distinguish between horizontal and vertical movement in SignWriting.\\
\bul I am able to write and sign the Index Middle Ring handshape.\\
\bul I am able to write and sign the Index Middle Ring on Circle handshape.\\
\bul I am able to write and sign the Index Middle Ring, Bent handshape.\\
\bul I am able to write and sign the Index Middle Ring, Unit handshape.\\
\bul I am able to write and sign the Index Middle Ring, Unit Hinge handshape.\\
\bul I am able to write and sign the Baby Up handshape.\\
\bul I am able to draw nose.\\
\bul I am able to recognize the vocabulary for this lesson.\\
\bul I am able to read the practice sentences for this lesson.\\
\bul I am able to read the practice story for this lesson.\\
\end{tabular}

\subsection{Symbol Group Curve Floor}

The nineteenth Symbol Group we informally call curve floor.
Its official name is ``Curves Parallel Floor Plane'', but you can now list off the symbol groups as ``one, two, three, four, five, six, seven, eight, nine, thumb, contact, fingers, wall, diagonal, floor, curve wall, hit wall, hit floor, curve floor''.

\begin{center}
\begin{tabular}{rcrc}
\textbf{Base Symbol}&\textbf{Example}&\textbf{Base Symbol}&\textbf{Example}\\
Curve Floor Plane Small         &B511x505S2d500489x496&Curve Floor Plane Medium 1 &B515x506S2d600486x494\\
Curve Floor Plane Medium 2      &B520x507S2d700481x493&Curve Floor Plane Large    &B523x508S2d800477x493\\
Curve Floor Plane Combined      &B519x510S2d900481x490&Hump Floor Plane Small     &B520x506S2da00480x495\\
Loop Floor Plane Small          &B520x506S2db00480x495&Wave Floor Plane Snake     &B525x507S2dc00476x494\\
Wave Floor Plane Small          &B521x508S2dd00479x492&Wave Floor Plane Large     &B525x511S2de00475x490\\
Rotation Single Floor Plane     &B511x511S2df00490x490&Rotation Double Floor Plane&B511x515S2e000490x486\\
Rotation Alternating Floor Plane&B512x515S2e100489x486&Shaking Parallel Floor     &B510x516S2e200490x484\\
\end{tabular}
aning is \begin{center}
\begin{tabular}{r*{6}{c}}
&\textbf{Fill 1}&\textbf{Fill 2}&\textbf{Fill 3}&\textbf{Fill 4}&\textbf{Fill 5}&\textbf{Fill 6}\\
\begin{tabular}{p{1cm}p{14cm}}
\bul I am able to draw the heel fist palmshape in all forms.\\
\bul I know which base symbols are in Symbol Group curve floor.\\
\bul I know which base symbols are in Symbol Group circles.\\
\bul I am able to distinguish between horizontal and vertical movement in SignWriting.\\
\bul I am able to write and sign the Index Middle Ring handshape.\\
\bul I am able to write and sign the Index Middle Ring on Circle handshape.\\
\bul I am able to write and sign the Index Middle Ring, Bent handshape.\\
\bul I am able to write and sign the Index Middle Ring, Unit handshape.\\
\bul I am able to write and sign the Index Middle Ring, Unit Hinge handshape.\\
\bul I am able to write and sign the Baby Up handshape.\\
\bul I am able to draw nose.\\
\bul I am able to recognize the vocabulary for this lesson.\\
\bul I am able to read the practice sentences for this lesson.\\
\bul I am able to read the practice story for this lesson.\\
\end{tabular}
\begin{center}\textbf{\Huge This is where we are!}\end{center}\end{document}
