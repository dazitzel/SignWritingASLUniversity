\documentclass{article}

\usepackage{fontspec}
\usepackage{fullpage}
\usepackage{multicol}
\usepackage{multirow}
\usepackage{tikz}

\begin{document}

\newfontfamily\swfill{SuttonSignWritingFill.ttf}
\newfontfamily\swline{SuttonSignWritingLine.ttf}
\newcommand{\bul}{\hfil$\bullet$&}
\renewenvironment{glossary}{\begin{multicols}{5}\begin{center}}{\end{center}\end{multicols}}
\setcounter{secnumdepth}{0}
\setlength{\columnseprule}{1pt}

\section{Supplement For Lesson 13}

\subsection{Objectives}

\begin{tabular}{p{1cm}p{14cm}}
\bul I have completed the objectives for this lesson.\\
\bul I know which base symbols are in Symbol Group lips.\\
\bul I know which base symbols are in Symbol Group mouth.\\
\bul I am able to write and sign the Middle Ring Baby on Circle handshape.\\
\bul I am able to write and sign the Middle Ring Baby on Angle handshape.\\
\bul I am able to write and sign the Index Thumb Side handshape.\\
\bul I am able to write and sign the Index Thumb Side, Thumb Bent handshape.\\
\bul I am able to write and sign the Index Thumb Side, Index Bent handshape.\\
\bul I am able to write and sign the Index Thumb Hook handshape.\\
\bul I am able to write and sign the Index Thumb Curve, Thumb Under handshape.\\
\bul I am able to draw teeth.\\
\bul I am able to recognize the vocabulary for this lesson.\\
\bul I am able to read the practice sentences for this lesson.\\
\bul I am able to read the practice story for this lesson.\\
\end{tabular}

\subsection{Symbol Group Lips}

The twenty-fifth Symbol Group we informally call lips.
Its official name is ``Mouth Lips'', but you can now list off the symbol groups as ``one, two, three, four, five, six, seven, eight, nine, thumb, contact, fingers, wall, diagonal, floor, curve wall, hit wall, hit floor, curve floor, circles, timing, head, eyes, middle, lips''.

\begin{center}
\begin{tabular}{rcrc}
\textbf{Base Symbol}&\textbf{Example}&\textbf{Base Symbol}&\textbf{Example}\\
Mouth Closed Neutral         &B518x518S33b00482x483&Mouth Closed Forward     &B518x518S33c00482x483\\
Mouth Closed Contact         &B518x518S33d00482x483&Mouth Smile              &B518x518S33e00482x483\\
Mouth Smile Wrinkled         &B518x518S33f00482x483&Mouth Smile Open         &B518x518S34000482x483\\
Mouth Frown                  &B518x518S34100482x483&Mouth Frown Wrinkled     &B518x518S34200482x483\\
Mouth Frown Open             &B518x518S34300482x483&Mouth Open Circle        &B518x518S34400482x483\\
Mouth Open Forward           &B518x518S34500482x483&Mouth Open Wrinkled      &B518x518S34600482x483\\
Mouth Open Oval              &B518x518S34700482x483&Mouth Open Oval Wrinkled &B518x518S34800482x483\\
Mouth Open Oval Yawn         &B518x518S34900482x483&Mouth Open Rectangle     &B518x518S34a00482x483\\
Mouth Open Rectangle Wrinkled&B518x518S34b00482x483&Mouth Open Rectangle Yawn&B518x518S34c00482x483\\
Mouth Kiss                   &B518x518S34d00482x483&Mouth Kiss Forward       &B518x518S34e00482x483\\
Mouth Kiss Wrinkled          &B518x518S34f00482x483&Mouth Tense              &B518x518S35000482x483\\
Mouth Tense Forward          &B518x518S35100482x483&Mouth Tense Sucked       &B518x518S35200482x483\\
Lips Pressed Together        &B518x518S35300482x483&Lip Lower Over Upper     &B518x518S35400482x483\\
Lip Upper Over Lower         &B518x518S35500482x483&Mouth Corners            &B518x518S35600482x483\\
Mouth Wrinkles Single        &B518x518S35700482x483&Mouth Wrinkles Double    &B518x518S35800482x483\\
\end{tabular}
\end{center}

\subsection{Symbol Group Mouth}

The twenty-sixth Symbol Group we informally call mouth.
Its official name is ``Tongue Teeth Chin Neck'', but you can now list off the symbol groups as ``one, two, three, four, five, six, seven, eight, nine, thumb, contact, fingers, wall, diagonal, floor, curve wall, hit wall, hit floor, curve floor, circles, timing, head, eyes, middle, lips, mouth''.

\begin{center}
\begin{tabular}{rcrc}
\textbf{Base Symbol}&\textbf{Example}&\textbf{Base Symbol}&\textbf{Example}\\
Tongue Sticks Out Far      &B518x518S35900482x483&Tongue Licks Lips              &B518x518S35a00482x483\\
Tongue Tip Between Lips    &B518x518S35b00482x483&Tongue Tip Touches Inside Mouth&B518x518S35c00482x483\\
Tongue Inside Mouth Relaxed&B518x518S35d00482x483&Tongue Moves Against Cheek     &B520x518S35e00481x483\\
Tongue Center Sticks Out   &B518x518S35f00482x483&Tongue Center Inside Mouth     &B518x518S36000482x483\\
Teeth                      &B518x518S36100482x483&Teeth Movement                 &B518x518S36200482x483\\
Teeth on Tongue            &B518x518S36300482x483&Teeth on Tongue Movement       &B518x518S36400482x483\\
Teeth on Lips              &B518x518S36500482x483&Teeth on Lips Movement         &B518x518S36600482x483\\
Teeth Bite Lips            &B518x518S36700482x483&Jaw Movement Wall Plane        &B521x519S36800479x481\\
Jaw Movement Floor Plane   &B520x519S36900480x481&Neck                           &B518x523S36a00482x477\\
Hair                       &B518x518S36b00482x482&Excitement                     &B524x524S36c00477x477\\
\end{tabular}
\end{center}

\begin{tabular}{p{1cm}p{14cm}}
\bul I know which base symbols are in Symbol Group mouth.\\
\bul I am able to write and sign the Middle Ring Baby on Circle handshape.\\
\bul I am able to write and sign the Middle Ring Baby on Angle handshape.\\
\bul I am able to write and sign the Index Thumb Side handshape.\\
\bul I am able to write and sign the Index Thumb Side, Thumb Bent handshape.\\
\bul I am able to write and sign the Index Thumb Side, Index Bent handshape.\\
\bul I am able to write and sign the Index Thumb Hook handshape.\\
\bul I am able to write and sign the Index Thumb Curve, Thumb Under handshape.\\
\bul I am able to draw teeth.\\
\bul I am able to recognize the vocabulary for this lesson.\\
\bul I am able to read the practice sentences for this lesson.\\
\bul I am able to read the practice story for this lesson.\\
\end{tabular}
\begin{center}\textbf{\Huge This is where we are!}\end{center}\end{document}
