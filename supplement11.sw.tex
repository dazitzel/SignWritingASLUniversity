\documentclass{article}

\usepackage{fontspec}
\usepackage{fullpage}
\usepackage{multicol}
\usepackage{multirow}
\usepackage{tikz}

\begin{document}

\newfontfamily\swfill{SuttonSignWritingFill.ttf}
\newfontfamily\swline{SuttonSignWritingLine.ttf}
\newcommand{\bul}{\hfil$\bullet$&}
\renewenvironment{glossary}{\begin{multicols}{5}\begin{center}}{\end{center}\end{multicols}}
\setcounter{secnumdepth}{0}
\setlength{\columnseprule}{1pt}

\section{Supplement For Lesson 11}

\subsection{Objectives}

\begin{tabular}{p{1cm}p{14cm}}
\bul I have completed the objectives for this lesson.\\
\bul I know which base symbols are in Symbol Group timing.\\
\bul I know which base symbols are in Symbol Group head.\\
\bul I am able to write and sign the Baby Thumb handshape.\\
\bul I am able to write and sign the Baby Thumb on Hinge handshape.\\
\bul I am able to write and sign the Baby Index Thumb handshape.\\
\bul I am able to write and sign the Baby Index Thumb on Hinge handshape.\\
\bul I am able to write and sign the Baby Index handshape.\\
\bul I am able to draw mouth.\\
\bul I am able to recognize the vocabulary for this lesson.\\
\bul I am able to read the practice sentences for this lesson.\\
\bul I am able to read the practice story for this lesson.\\
\end{tabular}

\subsection{Symbol Group Timing}

The twenty-first Symbol Group we informally call timing.
It's official name is ``Dynamics \& Timing'', but you can now list off the symbol groups as ``one, two, three, four, five, six, seven, eight, nine, thumb, contact, fingers, wall, diagonal, floor, cerved wall, hit wall, hit floor, curved floor, circles, timing''.

\begin{center}
\begin{tabular}{rcrc}
\textbf{Base Symbol}&\textbf{Example}&\textbf{Base Symbol}&\textbf{Example}\\
Fast            &B506x504S2f700494x497&Slow                 &B519x506S2f800481x495\\
Tense           &B506x503S2f900495x498&Relaxed              &B506x504S2fa00495x497\\
Same Time       &B508x503S2fb00493x497&Same Time Alternating&B508x506S2fc00493x495\\
Every Other Time&B512x503S2fd00489x497&Gradual              &B508x507S2fe00493x493\\
\end{tabular}
\end{center}

\subsection{Symbol Group Head}

The twenty-second Symbol Group we informally call head.
It's official name is also ``Head'', but you can now list off the symbol groups as ``one, two, three, four, five, six, seven, eight, nine, thumb, contact, fingers, wall, diagonal, floor, cerved wall, hit wall, hit floor, curved floor, circles, timing, head''.

\begin{center}
\begin{tabular}{rcrc}
\textbf{Base Symbol}&\textbf{Example}&\textbf{Base Symbol}&\textbf{Example}\\
Head                                             &B518x518S2ff00482x483&Head Rims                                &B518x518S30000482x483\\
Head Movement Straight Wall Plane                &B518x523S30100482x477&Head Movement Tilts Wall Plane           &B518x523S30200482x478\\
Head Movement Straight Floor Plane               &B518x523S30300482x477&Head Movement Curves Wall Plane          &B518x522S30400482x478\\
Head Movement Curves Floor Plane                 &B518x521S30500482x479&Head Movement Circles                    &B518x519S30600482x481\\
Face Direction Positions, Nose Forward Tilting   &B518x518S30700482x483&Face Direction Positions, Nose Up or Down&B518x522S30800482x478\\
Face Direction Positions, Nose Up or Down Tilting&B518x522S30900482x479\\
\end{tabular}
\end{center}

\subsection{Baby Thumb Handshape}

The baby thumb handshape is made by extending the baby finger and thumb and keeping the other fingers held tightly.

\begin{tabular}{p{1cm}p{14cm}}
\bul I am able to write and sign the Baby Thumb handshape.\\
\bul I am able to write and sign the Baby Thumb on Hinge handshape.\\
\bul I am able to write and sign the Baby Index Thumb handshape.\\
\bul I am able to write and sign the Baby Index Thumb on Hinge handshape.\\
\bul I am able to write and sign the Baby Index handshape.\\
\bul I am able to draw mouth.\\
\bul I am able to recognize the vocabulary for this lesson.\\
\bul I am able to read the practice sentences for this lesson.\\
\bul I am able to read the practice story for this lesson.\\
\end{tabular}
\subsection{Baby Thumb on Hinge Handshape}

\subsection{Baby Index Thumb Handshape}

\subsection{Baby Index Thumb on Hinge Handshape}

\begin{center}\textbf{\Huge This is where we are!}\end{center}\end{document}
\subsection{Baby Index handshape}

