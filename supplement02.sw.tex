\documentclass{article}

\usepackage{fontspec}
\usepackage{fullpage}
\usepackage{multicol}
\usepackage{multirow}
\usepackage{tikz}

\begin{document}

\newfontfamily\swfill{SuttonSignWritingFill.ttf}
\newfontfamily\swline{SuttonSignWritingLine.ttf}
\newcommand{\bul}{\hfil$\bullet$&}
\renewenvironment{glossary}{\begin{multicols}{5}\begin{center}}{\end{center}\end{multicols}}
\setcounter{secnumdepth}{0}
\setlength{\columnseprule}{1pt}

\section{Supplement For Lesson 2}

\subsection{Objectives}

\begin{tabular}{p{1cm}p{14cm}}
\bul I have completed the objectives for this lesson.\\
\bul I am able to read each letter of the fingerspelled alphabet.\\
\bul I am able to read the numbers 6--10.\\
\bul I am able to list the seven categories of SignWrititng in order.\\
\bul I am able demonstrate the meaning and form of the symbol groups in the hands category in order.\\
\bul I know which base symbols are in Symbol Group one.\\
\bul I know which base symbols are in Symbol Group two.\\
\bul I know which base symbols are in Symbol Group three.\\
\bul I know what palmshape means.\\
\bul I am able to draw the Fist palmshape in all forms.\\
\bul I am able to draw and demonstrate what fill one means.\\
\bul I am able to draw and demonstrate what rotation means.\\
\bul I am able to write and sign the Index handshape.\\
\bul I am able to write and sign the Index on Circle handshape.\\
\bul I am able to write and sign the Index on Angle handshape.\\
\bul I am able to write and sign the Index Bent handshape.\\
\bul I am able to write and sign the Index Cup handshape.\\
\bul I am able to write and sign the touch movement.\\
\bul I am able to write the fast mark.\\
\bul I am able to write the head symbol.\\
\bul I am able to draw the shoulder/hip symbol.\\
\bul I am able to draw the comma punctuation mark.\\
\bul I am able to recognize the vocabulary for lesson.\\
\bul I am able to read the practice sentences for this lesson.\\
\bul I am able to read the practice stories for this lesson.\\
\end{tabular}

\subsection{The Fingerspelled Alphabet}

\begin{center}
\begin{tabular}{*{5}{c}}
\textbf{A}&\textbf{B}&\textbf{C}&\textbf{D}&\textbf{E}\\
B510x508S1f720490x493&B507x511S14720493x489&B509x510S16d20492x490&B508x515S10110492x485&B508x508S14a20493x493\\
\textbf{F}&\textbf{G}&\textbf{H}&\textbf{I}&\textbf{J}\\
B511x515S1ce20489x485&B515x508S1f000486x493&B515x508S11502485x493&B511x510S19220490x491&B519x519S19220498x500S2a20c481x482\\
\textbf{K}&\textbf{L}&\textbf{M}&\textbf{N}&\textbf{O}\\
B515x515S14020486x485&B512x515S1dc20488x485&B510x513S18d20490x488&B511x513S11920490x487&B508x508S17620492x492\\
\textbf{P}&\textbf{Q}&\textbf{R}&\textbf{S}&\textbf{T}\\
B516x512S14021485x488&B515x512S1f021485x489&B508x515S11a20493x485&B508x508S20320493x493&B508x510S1fb20493x491\\
&\textbf{U}&\textbf{V}&\textbf{W}\\
&B508x515S11520493x485&B508x515S10e20493x485&B509x515S18720491x486\\
&\textbf{X}&\textbf{Y}&\textbf{Z}\\
&B511x513S10620490x487&B514x510S19a20486x490&B519x518S10020481x488S2450a488x483\\
\end{tabular}
\end{center}

\subsection{The Numbers 6--10}

\begin{center}
\begin{tabular}{*{5}{c}}
\textbf{Six}&\textbf{Seven}&\textbf{Eight}&\textbf{Nine}&\textbf{Ten}\\
B509x515S18720491x486&B511x514S1a520490x486&B511x514S1bb20490x486&B511x515S1ce20489x485&B513x528S1f540488x504S2a538494x472\\
\end{tabular}
\end{center}

\subsection{SignWriting Categories}

Every SignWriting symbol has been assigned to one of seven categories, and each category has between one and ten Symbol Groups within it.
This is very similar to knowing whether a letter in English in a consonant or vowel, it's just that SignWriting has a larger vocabulary of groups of base symbols.

In order to be considered literate you will be able to know which categories each base symbol is in.
Because the base symbols of SignWriting are organized by category, if you can do the list of Symbol Groups in order then you can also list the categories in order.

\begin{center}
\begin{tabular}{ccc@{\hskip 2cm}ccc}
\textbf{Category}&\textbf{Example}&\textbf{Name}&\textbf{Category}&\textbf{Example}&\textbf{Name}\\
\textbf{1}&B508x515S10000493x485&Hands      &\textbf{2}&B507x508S22a00494x493&Movement\\
\textbf{3}&B506x504S2f700494x497&Timing     &\textbf{4}&B518x518S30a00482x483&Face    \\
\textbf{5}&B523x506S36e00477x494&Body       &\textbf{6}&B521x521S37f00480x480&Detail     \\
\textbf{7}&B537x504S38700463x496&Punctuation\\
\end{tabular}
\end{center}

\subsection{The Hands Category}

You will eventually be able to recite the informal names of all thirty symbol groups in order as easily as you can recite the alphabet.
The symbol groups can be thought of a three groups of ten, and the first ten are extremely easy to remember.
The hands category consists of ten base symbols roughly based on the first ten numbers in ASL.

\begin{center}
\begin{tabular}{ccc@{\hskip 5mm}ccc}
\textbf{Symbol}&&&\textbf{Symbol}\\
\textbf{Group}&\textbf{Name}&\textbf{Example}&\textbf{Group}&\textbf{Name}&\textbf{Example}\\
\textbf{1}&One  &B508x515S10000493x485&\textbf{2} &Two  &B508x515S10e00493x485\\
\textbf{3}&Three&B512x515S11e00489x485&\textbf{4} &Four &B511x516S14400489x485\\
\textbf{5}&Five &B512x516S14c00489x485&\textbf{6} &Six  &B509x515S18600491x485\\
\textbf{7}&Seven&B510x515S1a400490x485&\textbf{8} &Eight&B510x515S1ba00490x485\\
\textbf{9}&Nine &B510x515S1cd00490x485&\textbf{10}&Thumb&B512x508S1f500488x493\\
\end{tabular}
\end{center}

It's important to remember that these names are informal.
B510x515S1cd00490x485 does not correspond to the ASL number nine.
Though these unofficial names are reminiscent.
For instance compare B510x515S1cd00490x485 with B511x515S1ce20489x485.
You will learn the official names but for learning SignWriting, we are just asking you to know the informal names in order.

On the plus side, the first nine you should be able to get in order without a whole lot of work.

\subsection{Symbol Group One}

The first Symbol Group we informally call one.
It's official name is Index, but for now be ready that if you are asked for a list of Symbol Groups it goes ``one''.
Symbol Group Index (One) consists of all handshapes where the index finger features prominently.
If there are other fingers, they are a group of fingers and secondary to the handshape.

The handshapes in symbol group one which are needed for ASL are:

\begin{center}
\begin{tabular}{rcrc}
\textbf{Base Symbol}&\textbf{Example}&\textbf{Base Symbol}&\textbf{Example}\\
Index         &B508x515S10000493x485&Index on Circle&B508x515S10110492x485\\
Index on Angle&B511x515S10410489x485&Index Bent     &B508x513S10610493x487\\
Index Cup     &B508x513S10a10493x487\\
\end{tabular}
\end{center}

And just to be perfectly clear, the order of these symbols is: Index, Index on Circle, Index on Angle, Index Bent, Index Cup.

\subsection{Symbol Group Two}

The second Symbol Group we informally call two.
It's official name is ``Index Middle'', but you should be able to list off the Symbol Groups as ``one and two''.
Symbol Group Index Middle (Two) consists of all handshapes with the index and middle finger extended and no other fingers extended.

The handshapes in symbol group two which are needed for ASL are:

\begin{center}
\begin{tabular}{rcrc}
\textbf{Base Symbol}&\textbf{Example}&\textbf{Base Symbol}&\textbf{Example}\\
Index Middle            &B508x515S10e00493x485&Index Middle Bent     &B508x514S11010493x487\\
Index Middle Unit       &B508x515S11510493x485&Index Middle Unit, Cup&B508x514S11810493x487\\
Index Middle Unit, Hinge&B509x513S11910492x488&Index Middle Cross    &B508x515S11a10493x485\\
\end{tabular}
\end{center}

\subsection{Symbol Group Three}

The third Symbol Group we informally call three.
It's official name is ``Index Middle Thumb'', but you should now be able to list off the Symbol Groups as ``one, two, three''.
Symbol Group Index Middle Thumb (Three) consists of all handshapes with the index, middle fingers, and thumb extended and no other fingers extended.

The handshapes in symbol group three which are needed for ASL are:

\begin{center}
\begin{tabular}{rcrc}
\textbf{Base Symbol}&\textbf{Example}&\textbf{Base Symbol}&\textbf{Example}\\
Index Middle Thumb            &B512x515S11e00489x485&Index Middle Bent, Thumb Straight  &B512x514S12110488x487\\
Index Middle Thumb Bent       &B511x514S12210489x487&Index Up, Middle Hinge, Thumb Side &B514x515S12410487x485\\
Index Middle Thumb Cup        &B512x512S12810489x488&Index Middle Thumb Circle          &B512x511S12910488x489\\
Index Middle Unit, Thumb Side &B512x515S12d10489x485&Index Middle Unit Hinge, Thumb Side&B512x513S13210488x488\\
Index Middle Cross, Thumb Side&B511x515S13310489x485&Middle Thumb Circle, Index Up      &B512x515S13810488x485\\
Index Middle Thumb, Angle     &B515x508S13f10486x493&Middle Thumb Angle Out, Index Up   &B515x515S14010486x485\\
\end{tabular}
\end{center}

By this point you have probably noticed that the first symbol has an empty center to it while the remaining ones have a partially filled in center.
The reason for this is primarily because the first shape identifies the symbol group and the remaining shapes are much more iconic in this form.
What this means will be exlpained a little later in this lesson,.

\subsection{Palmshapes}

Each hand shape has a palmshape at it's root, the shape you draw first before decorating it with fingers.
Each palmshape represents how tightly the ``missing'' fingers are held together.
They come in two primary types of fist and heel.

In the fist type are fist (meaning that the missing fingers are held tightly),
circle (meaning that the missing fingers are held loosely),
side (meaning that the fingers are open and held firm and not actually missing),
cup (meaning that the fingers are open and relaxed and not actually missing), and
flat (meaning that the hand is all the way open).

In the heel type are fist (meaninng that the missing fingers are held tightly),
and flat (meaning that the hand is all the way open).

The heel palmshapes are technically redundant, but are important in correctly expressing meaning by (and to) someone who is fluent in ASL.
As an example: point your fingers straight forward, open your hand and pretend you are placing your palm on an imaginary table with your palm pointing straight down.
This can be either B512x516S14c50488x485 or B515x509S14d14485x491.
Both are correct recordings of this handshape and position, but (depending on the word and context) one will be the most correct.
If looking up an unfamiliar sign it may help to consider both options.

We will spend some time on each of these palmshapes, but here are examples of each so you know what you are looking at.

\begin{center}
\begin{tabular}{ccccc}
\textbf{Fist}     &B500x500S20300500x500&&\textbf{Circle}   &B500x500S17600500x500\\
\textbf{Side}     &B500x500S16800500x500&&\textbf{Cup}      &B500x500S17100500x500\\
\textbf{Flat}     &B500x500S15a00500x500\\
\textbf{Fist Heel}&B500x500S20410500x500&&\textbf{Flat Heel}&B500x500S15c10500x500\\
\end{tabular}
\end{center}

\subsection{The Fist Palmshape}

Yes, I am aware that we just talked about this but $\ldots$ the fist palmshape represents holding the hand tightly.

\begin{center}
\begin{tabular}{r*{6}{c}}
&\textbf{Fill 1}&\textbf{Fill 2}&\textbf{Fill 3}&\textbf{Fill 4}&\textbf{Fill 5}&\textbf{Fill 6}\\
\multirow{2}{*}{\textbf{Right Hand}}&
B500x500S20300500x500&
B500x500S20310500x500&
B500x500S20320500x500&
B500x500S20330500x500&
B500x500S20340500x500&
B500x500S20350500x500\\
&
\tikz{\draw[thick](0,0)rectangle(10pt,10pt);}&
\tikz{\draw[thick](0,0)rectangle(10pt,10pt);\draw[thick](5pt,10pt)--(5pt,0);\draw[thick](10pt,10pt)--(5pt,0);\draw[thick](5pt,10pt)--(10pt,0);}&
\tikz{\draw[thick](0,0)rectangle(10pt,10pt);\draw[thick](0,0)--(10pt,10pt);\draw[thick](0,10pt)--(10pt,0);}&
\tikz{\draw[thick](0,0)rectangle(10pt,10pt);\draw[thick](-3pt,7pt)--(13pt,7pt);}&
\tikz{\draw[thick](0,0)rectangle(10pt,10pt);\draw[thick](5pt,10pt)--(5pt,0);\draw[thick](10pt,10pt)--(5pt,0);\draw[thick](5pt,10pt)--(10pt,0);\draw[thick](-3pt,7pt)--(13pt,7pt);}&
\tikz{\draw[thick](0,0)rectangle(10pt,10pt);\draw[thick](0,0)--(10pt,10pt);\draw[thick](0,10pt)--(10pt,0);\draw[thick](-3pt,7pt)--(13pt,7pt);}\\
\multirow{2}{*}{\textbf{Left Hand}}&
B500x500S20308500x500&
B500x500S20318500x500&
B500x500S20328500x500&
B500x500S20338500x500&
B500x500S20348500x500&
B500x500S20358500x500\\
&
\tikz{\draw[thick](0,0)rectangle(10pt,10pt);}&
\tikz{\draw[thick](0,0)rectangle(10pt,10pt);\draw[thick](5pt,10pt)--(5pt,0);\draw[thick](0,10pt)--(5pt,0);\draw[thick](0,0)--(5pt,10pt);}&
\tikz{\draw[thick](0,0)rectangle(10pt,10pt);\draw[thick](0,0)--(10pt,10pt);\draw[thick](0,10pt)--(10pt,0);}&
\tikz{\draw[thick](0,0)rectangle(10pt,10pt);\draw[thick](-3pt,7pt)--(13pt,7pt);}&
\tikz{\draw[thick](0,0)rectangle(10pt,10pt);\draw[thick](5pt,10pt)--(5pt,0);\draw[thick](0,10pt)--(5pt,0);\draw[thick](0,0)--(5pt,10pt);\draw[thick](-3pt,7pt)--(13pt,7pt);}&
\tikz{\draw[thick](0,0)rectangle(10pt,10pt);\draw[thick](0,0)--(10pt,10pt);\draw[thick](0,10pt)--(10pt,0);\draw[thick](-3pt,7pt)--(13pt,7pt);}\\
\end{tabular}
\end{center}

The first row is what the right hand will look like in most of this manual.
The second row is an approximation of what it will look like when you write it.
The third row is what the left hand will look like in most of this manual.
The fourth row is an approximation of what it will look like when you write it.

When writing fill 1 (for either hand), simply draw a square.
For fills 2 and 5: draw a square, draw a line down the middle, and then draw an ``X'' through the correct half to indicate the shading.
For fillls 3 and 6: draw a rectangle, and then draw an ``X'' in the square to indicate full shading.

For fills 4--6, add a line through the top portion of the glyph which sticks out on both ends;
though for handshapes with fingers, you will draw the line through the fingers instead.

\subsection{The First Fill}

\noindent
\textbf{Hand Symbols}

Any symbol drawn in the first fill means that the signer's palm is facing the signer.
For all the hand symbols, the empty portion represents the signer's palm and the filled portion represents the back of the hand.
So for fill one, if the hand was open you would be able to see all of your palm and none of the back of your hand --- leaving fill one an empty symbol.

\noindent
\textbf{Movement Symbols}

Any symbol drawn in the first fill means that the right hand is doing the movement, so B507x508S22a00494x493 means that the right hand is moving up.

\noindent
\textbf{Everything Else}

The fills for other categories tend to be a bit more iconic.
We leave the discovery of the meaning to the student as they encounter them.

\subsection{Rotation}

In some cases, rotation does not provides semantic information so only one rotation exists:
B518x518S30a00482x483

For most symbols, rotation is simply one of eight directions:
B507x508S22a00494x493 B507x507S22a01494x493 B508x507S22a02493x494 B507x507S22a03494x493 B507x508S22a04494x493 B507x507S22a05494x493 B508x507S22a06493x494 B507x507S22a07494x493

And some symbols have mirroring:
B508x515S10000493x485 B511x515S10007490x485 B515x508S10006485x493 B515x511S10005485x490 B508x515S10004493x485 B511x515S10003490x485 B515x508S10002485x493 B515x511S10001485x490
B508x515S10008493x485 B511x515S10009490x485 B515x508S1000a485x493 B515x511S1000b485x490 B508x515S1000c493x485 B511x515S1000d490x485 B515x508S1000e485x493 B515x511S1000f485x490

For symbols that represent something parallel to the wall, the symbol represents the direction if you were holding the paper parallel to the wall in front of you.
For instance, {\small B508x515S10000493x485} means your index finger is pointing up while {\small B515x508S10002485x493} means that your index finger is pointing to your left.

For symbols that represent something parallel to the floor, the symbol represents the direction if you set the paper on the floor in front of you.
For instance, {\small B508x515S10030493x485} means your index finger is pointing forward, {\small B515x508S10032485x493} means that your index finger is pointing to your left, and {\small B508x515S10034493x485} means your index finger is pointing back toward yourself.

\subsection{The Index Handshape}

The Index handshape means that your index is extended straight and the rest of your fingers are held tightly, much as if you were pointing with one finger.

\begin{center}
\begin{tabular}{r*{6}{c}}
&\textbf{Fill 1}&\textbf{Fill 2}&\textbf{Fill 3}&\textbf{Fill 4}&\textbf{Fill 5}&\textbf{Fill 6}\\
\multirow{2}{*}{\textbf{Right Hand}}&
B508x515S10000493x485&
B508x515S10010493x485&
B508x515S10020493x485&
B508x515S10030493x485&
B508x515S10040493x485&
B508x515S10050493x485\\
&
\tikz{\draw[thick](0,0)rectangle(10pt,10pt);\draw[thick](10pt,20pt)--(10pt,0);}&
\tikz{\draw[thick](0,0)rectangle(10pt,10pt);\draw[thick](10pt,20pt)--(10pt,0);\draw[thick](5pt,10pt)--(5pt,0);\draw[thick](10pt,10pt)--(5pt,0);\draw[thick](5pt,10pt)--(10pt,0);}&
\tikz{\draw[thick](0,0)rectangle(10pt,10pt);\draw[thick](0,20pt)--(0,0);\draw[thick](0,0)--(10pt,10pt);\draw[thick](0,10pt)--(10pt,0);}&
\tikz{\draw[thick](0,0)rectangle(10pt,10pt);\draw[thick](10pt,20pt)--(10pt,0);\draw[thick](5pt,15pt)--(13pt,15pt);}&
\tikz{\draw[thick](0,0)rectangle(10pt,10pt);\draw[thick](10pt,20pt)--(10pt,0);\draw[thick](5pt,10pt)--(5pt,0);\draw[thick](10pt,10pt)--(5pt,0);\draw[thick](5pt,10pt)--(10pt,0);\draw[thick](5pt,15pt)--(13pt,15pt);}&
\tikz{\draw[thick](0,0)rectangle(10pt,10pt);\draw[thick](0,20pt)--(0,0);\draw[thick](0,0)--(10pt,10pt);\draw[thick](0,10pt)--(10pt,0);\draw[thick](-3pt,15pt)--(5pt,15pt);}\\
\multirow{2}{*}{\textbf{Left Hand}}&
B508x515S10008493x485&
B508x515S10018493x485&
B508x515S10028493x485&
B508x515S10038493x485&
B508x515S10048493x485&
B508x515S10058493x485\\
&
\tikz{\draw[thick](0,0)rectangle(10pt,10pt);\draw[thick](0,20pt)--(0,0);}&
\tikz{\draw[thick](0,0)rectangle(10pt,10pt);\draw[thick](0,20pt)--(0,0);\draw[thick](5pt,10pt)--(5pt,0);\draw[thick](0,10pt)--(5pt,0);\draw[thick](0,0)--(5pt,10pt);}&
\tikz{\draw[thick](0,0)rectangle(10pt,10pt);\draw[thick](10pt,20pt)--(10pt,0);\draw[thick](0,0)--(10pt,10pt);\draw[thick](0,10pt)--(10pt,0);}&
\tikz{\draw[thick](0,0)rectangle(10pt,10pt);\draw[thick](0,20pt)--(0,0);\draw[thick](-3pt,15pt)--(5pt,15pt);}&
\tikz{\draw[thick](0,0)rectangle(10pt,10pt);\draw[thick](0,20pt)--(0,0);\draw[thick](5pt,10pt)--(5pt,0);\draw[thick](0,10pt)--(5pt,0);\draw[thick](0,0)--(5pt,10pt);\draw[thick](-3pt,15pt)--(5pt,15pt);}&
\tikz{\draw[thick](0,0)rectangle(10pt,10pt);\draw[thick](10pt,20pt)--(10pt,0);\draw[thick](0,0)--(10pt,10pt);\draw[thick](0,10pt)--(10pt,0);\draw[thick](5pt,15pt)--(13pt,15pt);}\\
\end{tabular}
\end{center}

When writing fill 1: draw a square, then add the index finger.
For fills 2 and 5: draw a square, draw a line down the middle, draw an ``X'' through the correct half to indicate the shading, then add the index finger.
For fillls 3 and 6: draw a rectangle, draw an ``X'' in the square to indicate full shading, then add the index finger.

For fills 4--6, after you have drawn the index finger, add a line through it.

\subsection{Index on Circle}

The Index on Circle handshape means that your index finger is extended straight and the rest of your fingers are held loosely with the thumb supporting the remaining fingers in a circle.

\begin{center}
\begin{tabular}{r*{6}{c}}
&\textbf{Fill 1}&\textbf{Fill 2}&\textbf{Fill 3}&\textbf{Fill 4}&\textbf{Fill 5}&\textbf{Fill 6}\\
\multirow{2}{*}{\textbf{Right Hand}}&
B508x515S10100493x485&
B508x515S10110493x485&
B508x515S10120493x485&
B508x515S10130493x485&
B508x515S10140493x485&
B508x515S10150493x485\\
&
\tikz{\draw[thick](5pt,5pt)circle(5pt);\draw[thick](10pt,20pt)--(10pt,5pt);}&
\tikz{\draw[thick](5pt,5pt)circle(5pt);\draw[thick](10pt,20pt)--(10pt,5pt);\draw[thick](5pt,10pt)--(5pt,0);\draw[thick](8pt,2pt)--(5pt,10pt);\draw[thick](8pt,8pt)--(5pt,0);}&
\tikz{\draw[thick](5pt,5pt)circle(5pt);\draw[thick](0,20pt)--(0,5pt);\draw[thick](2pt,2pt)--(8pt,8pt);\draw[thick](2pt,8pt)--(8pt,2pt);}&
\tikz{\draw[thick](5pt,5pt)circle(5pt);\draw[thick](10pt,20pt)--(10pt,5pt);\draw[thick](5pt,15pt)--(13pt,15pt);}&
\tikz{\draw[thick](5pt,5pt)circle(5pt);\draw[thick](10pt,20pt)--(10pt,5pt);\draw[thick](5pt,10pt)--(5pt,0);\draw[thick](8pt,2pt)--(5pt,10pt);\draw[thick](8pt,8pt)--(5pt,0);\draw[thick](5pt,15pt)--(13pt,15pt);}&
\tikz{\draw[thick](5pt,5pt)circle(5pt);\draw[thick](0,20pt)--(0,5pt);\draw[thick](2pt,2pt)--(8pt,8pt);\draw[thick](2pt,8pt)--(8pt,2pt);\draw[thick](-3pt,15pt)--(5pt,15pt);}\\
\multirow{2}{*}{\textbf{Left Hand}}&
B508x515S10108493x485&
B508x515S10118493x485&
B508x515S10128493x485&
B508x515S10138493x485&
B508x515S10148493x485&
B508x515S10158493x485\\
&
\tikz{\draw[thick](5pt,5pt)circle(5pt);\draw[thick](0,20pt)--(0,5pt);}&
\tikz{\draw[thick](5pt,5pt)circle(5pt);\draw[thick](0,20pt)--(0,5pt);\draw[thick](5pt,10pt)--(5pt,0);\draw[thick](2pt,2pt)--(5pt,10pt);\draw[thick](2pt,8pt)--(5pt,0);}&
\tikz{\draw[thick](5pt,5pt)circle(5pt);\draw[thick](10pt,20pt)--(10pt,5pt);\draw[thick](2pt,2pt)--(8pt,8pt);\draw[thick](2pt,8pt)--(8pt,2pt);}&
\tikz{\draw[thick](5pt,5pt)circle(5pt);\draw[thick](0,20pt)--(0,5pt);\draw[thick](-3pt,15pt)--(5pt,15pt);}&
\tikz{\draw[thick](5pt,5pt)circle(5pt);\draw[thick](0,20pt)--(0,5pt);\draw[thick](5pt,10pt)--(5pt,0);\draw[thick](2pt,2pt)--(5pt,10pt);\draw[thick](2pt,8pt)--(5pt,0);\draw[thick](-3pt,15pt)--(5pt,15pt);}&
\tikz{\draw[thick](5pt,5pt)circle(5pt);\draw[thick](10pt,20pt)--(10pt,5pt);\draw[thick](2pt,2pt)--(8pt,8pt);\draw[thick](2pt,8pt)--(8pt,2pt);\draw[thick](5pt,15pt)--(13pt,15pt);}\\
\end{tabular}
\end{center}

As you practice to write these don't forget to practice rotations as well!

\subsection{Index on Angle}

The Index on Angle handshape means that your index finger is extended straight; your middle, ring, and baby fingers are held together extending the direction of your palm; and your thumb in brought up to touch the middle segment of your middle finger.

\begin{center}
\begin{tabular}{r*{6}{c}}
&\textbf{Fill 1}&\textbf{Fill 2}&\textbf{Fill 3}&\textbf{Fill 4}&\textbf{Fill 5}&\textbf{Fill 6}\\
\multirow{2}{*}{\textbf{Right Hand}}&
B508x515S10500493x485&
B508x515S10510493x485&
B508x515S10520493x485&
B508x515S10530493x485&
B508x515S10540493x485&
B508x515S10550493x485\\
&
\tikz{\draw[thick](0,0)rectangle(5pt,10pt);\draw[thick](5pt,20pt)--(5pt,0);\draw[thick](-10pt,10pt)--(0,10pt);\draw[thick](0,5pt)--(-7pt,10pt);}&
\tikz{\draw[thick](0,0)rectangle(5pt,10pt);\draw[thick](2.5pt,10pt)--(2.5pt,0);\draw[thick](2.5pt,10pt)--(5pt,0);\draw[thick](5pt,10pt)--(2.5pt,0);\draw[thick](5pt,20pt)--(5pt,0);\draw[thick](-10pt,10pt)--(0,10pt);\draw[thick](0,5pt)--(-7pt,10pt);}&
\tikz{\draw[thick](0,0)rectangle(5pt,10pt);\draw[thick](0,10pt)--(5pt,0);\draw[thick](5pt,10pt)--(0,0);\draw[thick](0,20pt)--(0,0);\draw[thick](-10pt,10pt)--(0,10pt);\draw[thick](0,5pt)--(-7pt,10pt);}&
\tikz{\draw[thick](0,0)rectangle(5pt,10pt);\draw[thick](5pt,20pt)--(5pt,0);\draw[thick](-10pt,10pt)--(0,10pt);\draw[thick](0,5pt)--(-7pt,10pt);\draw[thick](-10pt,5pt)--(10pt,20pt);}&
\tikz{\draw[thick](0,0)rectangle(5pt,10pt);\draw[thick](2.5pt,10pt)--(2.5pt,0);\draw[thick](2.5pt,10pt)--(5pt,0);\draw[thick](5pt,10pt)--(2.5pt,0);\draw[thick](5pt,20pt)--(5pt,0);\draw[thick](-10pt,10pt)--(0,10pt);\draw[thick](0,5pt)--(-7pt,10pt);\draw[thick](-10pt,5pt)--(10pt,20pt);}&
\tikz{\draw[thick](0,0)rectangle(5pt,10pt);\draw[thick](0,10pt)--(5pt,0);\draw[thick](5pt,10pt)--(0,0);\draw[thick](0,20pt)--(0,0);\draw[thick](-10pt,10pt)--(0,10pt);\draw[thick](0,5pt)--(-7pt,10pt);\draw[thick](-10pt,5pt)--(10pt,20pt);}\\
\textbf{Left Hand}&
B508x515S10508493x485&
B508x515S10518493x485&
B508x515S10528493x485&
B508x515S10538493x485&
B508x515S10548493x485&
B508x515S10558493x485\\
\end{tabular}
\end{center}

\subsection{Index Bent}

The Index Bent handshape means that your index finger is extended and then bend the two extreme knuckles fully.
The remaining fingers are held tightly in a fist.

\begin{center}
\begin{tabular}{r*{6}{c}}
&\textbf{Fill 1}&\textbf{Fill 2}&\textbf{Fill 3}&\textbf{Fill 4}&\textbf{Fill 5}&\textbf{Fill 6}\\
\textbf{Right Hand}&
B500x500S10600500x500&
B500x500S10610500x500&
B500x500S10620500x500&
B500x500S10630500x500&
B500x500S10640500x500&
B500x500S10650500x500\\
\textbf{Left Hand}&
B500x500S10608500x500&
B500x500S10618500x500&
B500x500S10628500x500&
B500x500S10638500x500&
B500x500S10648500x500&
B500x500S10658500x500\\
\end{tabular}
\end{center}

\subsection{Index Cup}

The Index Cup handshape means that your index finger is extended and then curl the index finger 90 degrees.

\begin{center}
\begin{tabular}{r*{6}{c}}
&\textbf{Fill 1}&\textbf{Fill 2}&\textbf{Fill 3}&\textbf{Fill 4}&\textbf{Fill 5}&\textbf{Fill 6}\\
\textbf{Right Hand}&
B500x500S10a00500x500&
B500x500S10a10500x500&
B500x500S10a20500x500&
B500x500S10a30500x500&
B500x500S10a40500x500&
B500x500S10a50500x500\\
\textbf{Left Hand}&
B500x500S10a08500x500&
B500x500S10a18500x500&
B500x500S10a28500x500&
B500x500S10a38500x500&
B500x500S10a48500x500&
B500x500S10a58500x500\\
\end{tabular}
\end{center}

\subsection{Touch Movement}

\begin{center}
\begin{tabular}{r*{2}{c}}
&\textbf{Touch}&\textbf{Touch Between}\\
\textbf{Single}&B505x506S20500496x495&B509x508S20700492x493\\
\textbf{multiple}&B511x506S20600489x495&B517x508S20710484x493\\
\end{tabular}
\end{center}

\subsection{The Fast Mark}

\begin{center}
B506x504S2f700494x497
\end{center}

\subsection{The Head Symbol}

\begin{center}
B518x518S2ff00482x483
\end{center}

\subsection{The Shoulder/Hip Symbol}

The shoulder and hip symbols are identical and usually identified by whether the hands are above or below them.
In cases where it matters an vertical spine can be added to clarify which is meant.

\begin{center}
\begin{tabular}{rc}
 \textbf{Shoulders or Hips}&B521x502S36d00479x498\\
         \textbf{Shoulders}&B521x516S36d02479x484\\
\textbf{Shoulders and Hips}&B521x516S36d12479x484\\
\end{tabular}
\end{center}

\subsection{The Comma}

The comma comes with four dynamic marks: normal, fast, slow, and tense.

\begin{center}
\begin{tabular}{*{4}{c}}
B537x504S38700463x496&B537x506S38710463x495&B537x505S38720463x495&B537x505S38730463x496\\
\end{tabular}
\end{center}

\subsection{Vocabulary}

\begin{glossary}

\textbf{6}\\
AS18720M509x515S18720491x486

\textbf{7}\\
AS1a520M511x514S1a520490x486

\textbf{8}\\
AS1bb20M511x514S1bb20490x486

\textbf{9}\\
AS1ce20M511x515S1ce20489x485

\textbf{10}\\
AS1f540S2a538M512x528S1f540489x504S2a538493x473

\textbf{a (letter)}\\
AS1f720M510x508S1f720490x493

\textbf{address}\\
AS1f502S1f50aS22a20M520x522S1f502505x498S1f50a480x498S22a20493x478

\textbf{alone}\\
AS10000S2e806M516x525S10000492x495S2e806484x475

\textbf{b}\\
AS14720M507x511S14720493x489

\textbf{boy}\\
AS18510S26500S22104S2ff00M545x522S26500524x507S18510520x489S22104527x476S2ff00482x483

\textbf{brother}\\
AS1dc51S1dc42S1dc4aS20500S22b03S2ff00M538x568S1dc51508x466S1dc4a490x544S1dc42464x526S20500475x553S22b03501x512S2ff00482x483

\textbf{c}\\
AS16d20M509x510S16d20492x490

\textbf{child}\\
AS15a50S22f04M513x524S15a50494x476S22f04488x510

\textbf{children}\\
AS15a50S22a04S2d508M522x525S15a50478x475S22a04483x510S2d508500x511

\textbf{d}\\
AS10110M508x515S10110492x485

\textbf{dad}\\
AS14c10S20500S2ff00M518x518S2ff00482x483S20500495x469S14c10468x453

\textbf{divorce}\\
AS10140S10148S28905S20500S2891dS2fb04M535x531S10140504x469S10148484x469S20500498x473S28905508x504S2891d466x504S2fb04494x525

\textbf{e}\\
AS14a20M508x508S14a20493x493

\textbf{f}\\
AS1ce20M511x515S1ce20489x485

\textbf{fingerspell}\\
AS14c50S26606S22520M529x523S14c50476x492S22520472x478S26606499x505

\textbf{g}\\
AS1f000M515x508S1f000486x493

\textbf{girl}\\
AS1f540S22a03S20e00S2ff00M525x554S1f540510x509S22a03486x540S20e00497x531S2ff00482x483

\textbf{grandma}\\
AS14c10S20500S2b901S2ff00M535x539S2ff00482x483S14c10473x508S2b901509x506S20500497x519

\textbf{grandpa}\\
AS14c10S20500S2b901S2ff00M535x518S2ff00482x483S14c10468x454S2b901509x457S20500496x468

\textbf{h}\\
AS11502M515x508S11502485x493

\textbf{have}\\
AS18041S18049S20500S20500M532x518S18049468x483S18041507x483S20500486x507S20500504x507

\textbf{hey}\\
AS14c50S23504M519x528S14c50490x472S23504482x510

\textbf{how}\\
AS16740S16748S20500S2c400M538x524S20500484x491S2c400498x501S16740491x477S16748462x477

\textbf{how many}\\
AS20330S20330S22a20S14c00S14c08M526x535S22a20494x501S14c08474x465S14c00503x465S20338478x520S20330508x520

\textbf{husband}\\
AS16d10S22b03S16d51S16d09S20800S2ff00M537x564S16d03474x544S16d51489x526S16d10520x483S2ff00482x483S22b03510x506S20800494x553

\textbf{i (letter)}\\
AS19220M511x510S19220490x491

\textbf{j}\\
AS19220S2a20cM519x519S19220498x500S2a20c481x482

\textbf{just}\\
AS10020S2ed09M512x522S2ed09493x479S10020489x492

\textbf{k}\\
AS14020M515x515S14020486x485

\textbf{l}\\
AS1dc20M512x515S1dc20488x485

\textbf{lady}\\
AS1f540S20e00S22a03S14c10S20f00S2c400S2ff00M546x582S22a03502x526S20e00513x513S14c10483x551S2ff00482x483S20f00514x567S1f540527x502S2c400506x541

\textbf{life}\\
AS1f502S1f50aS22a20M520x522S1f502505x498S1f50a480x498S22a20493x478

\textbf{live}\\
AS1f502S1f50aS22a20M520x522S1f502505x498S1f50a480x498S22a20493x478

\textbf{m}\\
AS18d20M510x513S18d20490x488

\textbf{male}\\
AS18510S26500S22104S2ff00M545x522S26500524x507S18510520x489S22104527x476S2ff00482x483

\textbf{man}\\
AS14c10S20500S22c04S20500S2ff00M518x549S2ff00482x483S22c04493x494S14c10467x459S20500495x470S20500495x538

\textbf{many}\\
AS20300S20300S22a24S14c30S14c38M531x536S20300475x464S20300513x464S14c30507x505S14c38470x505S22a24495x486

\textbf{marriage}\\
AS17107S16d21S22b03M528x528S22b03504x472S17107478x502S16d21472x508

\textbf{marry}\\
AS17107S16d21S22b03M528x528S22b03504x472S17107478x502S16d21472x508

\textbf{married}\\
AS17107S16d21S22b03M528x528S22b03504x472S17107478x502S16d21472x508

\textbf{mom}\\
AS14c10S20500S2ff00M518x535S2ff00482x483S20500494x520S14c10471x504

\textbf{n}\\
AS11920M511x513S11920490x487

\textbf{o}\\
AS17620M508x508S17620492x492

\textbf{only}\\
AS10020S2ed09M512x522S2ed09493x479S10020489x492

\textbf{p}\\
AS14021M516x512S14021485x488

\textbf{parents}\\
AS14c10S20500S20500S14c10S22a00S2ff00M518x541S2ff00482x483S14c10466x457S14c10465x510S20500494x471S20500494x519S22a00466x492

\textbf{q}\\
AS1f021M515x512S1f021485x489

\textbf{r}\\
AS11a20M508x515S11a20493x485

\textbf{s}\\
AS20320M508x508S20320493x493

\textbf{single}\\
AS10000S2e806M516x525S10000492x495S2e806484x475

\textbf{sister}\\
AS1dc50S22a03S1dc0aS1dc02S20500S2ff00M539x571S2ff00482x483S1dc50515x494S1dc0a500x547S1dc02474x530S20500487x557S22a03509x524

\textbf{someone}\\
AS10000S2e806M516x525S10000492x495S2e806484x475

\textbf{something}\\
AS10000S2e806M516x525S10000492x495S2e806484x475

\textbf{spell}\\
AS14c50S26606S22520M529x523S14c50476x492S22520472x478S26606499x505

\textbf{t}\\
AS1fb20M508x510S1fb20493x491

\textbf{u}\\
AS11520M508x515S11520493x485

\textbf{v}\\
AS10e20M508x515S10e20493x485

\textbf{w}\\
AS18720M509x515S18720491x486

\textbf{wife}\\
AS16d10S22a03S16d51S16d39S20800S2ff00M527x581S2ff00482x483S22a03506x535S16d51486x546S16d10510x513S20800498x568S16d39470x561

\textbf{woman}\\
AS14c27S22a04S14c10S20500S2ff00M530x583S2ff00482x483S14c27504x508S14c10486x552S20500512x563S22a04495x534

\textbf{work}\\
AS20350S2035aS20600M521x516S20350480x485S2035a490x501S20600499x486

\textbf{x}\\
AS10620M511x513S10620490x487

\textbf{y}\\
AS19a20M514x510S19a20486x490

\textbf{z}\\
AS10020S2450aM519x518S10020481x488S2450a488x483

\end{glossary}

\subsection{Practice Sheet 2.A}

\begin{multicols}{5}
\begin{center}

M508x515S10000493x485 % 1
M536x504S38800464x496 % .
M519x528S14c50490x472S23504482x510 % hey
M537x504S38700463x496 % ,
M510x523S10040495x493S26500491x478 % you
M518x518S30c00482x483 % \?
M522x525S11541498x491S11549479x498S20600489x476 % name
M536x507S38900464x493 % ?
\vfil
\columnbreak

M508x515S10e00493x485 % 2
M536x504S38800464x496 % .
M528x528S22b03504x472S17107478x502S16d21472x508 % marry
M518x518S30a00482x483 % y/n
M510x523S10040495x493S26500491x478 % you
M536x507S38900464x493 % ?
\vfil
\columnbreak

M512x515S11e00489x485 % 3
M536x504S38800464x496 % .
M518x518S30a00482x483 % y/n
M522x525S15a50478x475S22a04483x510S2d508500x511 % children
M510x523S10040495x493S26500491x478 % you
M536x507S38900464x493 % ?
\vfil
\columnbreak

M511x516S14400489x485 % 4
M536x504S38800464x496 % .
M539x571S2ff00482x483S1dc50515x494S1dc0a500x547S1dc02474x530S20500487x557S22a03509x524 % sister
M510x523S10040495x493S26500491x478 % you
M518x518S30c00482x483 % \?
M526x535S22a20494x501S14c08474x465S14c00503x465S20338478x520S20330508x520 % how many
M536x507S38900464x493 % ?
\vfil
\columnbreak

M512x516S14c00489x485 % 5
M536x504S38800464x496 % .
M507x523S15a28494x496S26500493x477 % your
M518x535S2ff00482x483S20500494x520S14c10471x504 % mom
M518x518S30c00482x483 % \?
M522x525S11541498x491S11549479x498S20600489x476 % name
M536x507S38900464x493 % ?
\vfil

\end{center}
\end{multicols}

\subsection{Practice Sheet 2.B}

\begin{multicols}{5}
\begin{center}
M509x515S18720491x486 % 6
M536x504S38800464x496 % .
M518x518S30a00482x483 % y/n
M507x523S15a28494x496S26500493x477 % your
M518x518S2ff00482x483S20500495x469S14c10468x453 % dad
M544x531S20500509x515S10011523x501S20500520x488S2ff00482x483 % deaf
M536x507S38900464x493 % ?
\vfil
\columnbreak

M511x514S1a520490x486 % 7
M536x504S38800464x496 % .
M510x523S10040495x493S26500491x478 % you
M521x516S20350480x485S2035a490x501S20600499x486 % work
M518x518S30c00482x483 % \?
M518x525S10020482x476S27106503x485 % where
M536x507S38900464x493 % ?
\vfil
\columnbreak

M511x514S1bb20490x486 % 8
M536x504S38800464x496 % .
M510x523S10040495x493S26500491x478 % you
M520x522S1f502505x498S1f50a480x498S22a20493x478 % live
M518x518S30c00482x483 % \?
M518x525S10020482x476S27106503x485 % where
M536x507S38900464x493 % ?
\vfil
\columnbreak

M511x515S1ce20489x485 % 9
M536x504S38800464x496 % .
M515x519S10047485x498S26507501x481 % 3rd person
M518x518S30a00482x483 % y/n
M513x520S15a20487x493S26507500x481 % his / hers / its
M536x507S38900464x493 % ?
\vfil
\columnbreak

M513x528S2a538494x472S1f540488x504 % 10
M536x504S38800464x496 % .
M509x515S18720491x486 % w
M508x508S14a20493x493 % e
M518x518S30c00482x483 % \?
M523x535S2ea48483x510S10011502x466S2ea04508x500S10019477x475 % sign (as in ``signing'')
M536x507S38900464x493 % ?
\vfil

\end{center}
\end{multicols}

\subsection{Practice Sheet 2.C}

\begin{multicols}{5}
\begin{center}
M512x520S10000489x490S21d00494x480 % 11
M536x504S38800464x496 % .
M510x523S10040495x493S26500491x478 % you
M522x525S11541498x491S11549479x498S20600489x476 % name
B507x511S14720493x489 % b
B508x508S17620492x492 % o
B507x511S14720493x489 % b
M537x504S38700463x496 % ,
M518x518S30a00482x483 % y/n
M510x523S10040495x493S26500491x478 % you
M536x507S38900464x493 % ?
\vfil
\columnbreak

M509x521S10e00491x491S21d00491x480 % 12
M536x504S38800464x496 % .
M510x523S10040495x493S26500491x478 % you
M535x531S10140504x469S10148484x469S20500498x473S28905508x504S2891d466x504S2fb04494x525 % divorce
M518x518S30a00482x483 % y/n
M510x523S10040495x493S26500491x478 % you
M536x507S38900464x493 % ?
\vfil
\columnbreak

M513x519S22114487x481S12d00489x489 % 13
M536x504S38800464x496 % .
M538x568S1dc51508x466S1dc4a490x544S1dc42464x526S20500475x553S22b03501x512S2ff00482x483 % brother
M510x523S10040495x493S26500491x478 % you
M518x518S30c00482x483 % \?
M526x535S22a20494x501S14c08474x465S14c00503x465S20338478x520S20330508x520 % how many
M536x507S38900464x493 % ?
\vfil
\columnbreak

M513x515S14700493x493S22114487x486 % 14
M536x504S38800464x496 % .
M518x518S30a00482x483 % y/n
M510x523S10040495x493S26500491x478 % you
M532x518S18049468x483S18041507x483S20500486x507S20500504x507 % have
M539x571S2ff00482x483S1dc50515x494S1dc0a500x547S1dc02474x530S20500487x557S22a03509x524 % sister
M536x507S38900464x493 % ?
\vfil
\columnbreak

M513x518S22114487x483S15d00494x491 % 15
M536x504S38800464x496 % .
M507x523S15a28494x496S26500493x477 % your
M518x518S2ff00482x483S20500495x469S14c10468x453 % dad
M522x525S11541498x491S11549479x498S20600489x476 % name
M537x504S38700463x496 % ,
M518x518S30c00482x483 % \?
M529x523S14c50476x492S22520472x478S26606499x505 % fingerspell
M523x524S26503497x511S21100490x497S15a57477x476S15a51500x481
M536x504S38800464x496 % .
\vfil

\end{center}
\end{multicols}

\subsection{Practice Sheet 2.D}

\begin{multicols}{5}
\begin{center}
M520x522S18700502x492S2e00e480x479 % 16
M536x504S38800464x496 % .
M514x536S10008487x464S10020493x471S20500505x469S26600492x506 % you meet
M513x514S15a01490x486S20500487x503 % my
M538x568S1dc51508x466S1dc4a490x544S1dc42464x526S20500475x553S22b03501x512S2ff00482x483 % brother
M518x518S30a00482x483 % y/n
M510x523S10040495x493S26500491x478 % you
M536x507S38900464x493 % ?
\vfil
\columnbreak

M522x522S1a500501x494S2e00e478x478 % 17
M536x504S38800464x496 % .
M507x523S15a28494x496S26500493x477 % your
M527x581S2ff00482x483S22a03506x535S16d51486x546S16d10510x513S20800498x568S16d39470x561 % wife
M518x518S30c00482x483 % \?
M522x525S11541498x491S11549479x498S20600489x476 % name
M536x507S38900464x493 % ?
\vfil
\columnbreak

M523x522S1bb00502x492S2e00e478x479 % 18
M536x504S38800464x496 % .
M507x523S15a28494x496S26500493x477 % your
M539x571S2ff00482x483S1dc50515x494S1dc0a500x547S1dc02474x530S20500487x557S22a03509x524 % sister
M518x518S30a00482x483 % y/n
M516x525S10000492x495S2e806484x475 % single
M536x507S38900464x493 % ?
\vfil
\columnbreak

M524x522S1ce00502x490S2e00e477x479 % 19
M536x504S38800464x496 % .
M535x539S2ff00482x483S14c10473x508S2b901509x506S20500497x519 % grandma
M518x518S30c00482x483 % \?
M522x525S11541498x491S11549479x498S20600489x476 % name
M536x507S38900464x493 % ?
\vfil
\columnbreak

M517x513S22114484x488S1f420488x498 % 20
M536x504S38800464x496 % .
M508x510S1fb20493x491 % t
M515x508S11502485x493 % h
M508x508S14a20493x493 % e
M514x510S19a20486x490 % y
M518x518S30c00482x483 % \?
M523x535S2ea48483x510S10011502x466S2ea04508x500S10019477x475 % sign (as in ``signing'')
M536x507S38900464x493 % ?
\end{center}
\end{multicols}

\subsection{Story 2.A}

\begin{multicols}{5}
\begin{center}
M547x518S2ff00482x483S22a07534x476S14711519x487 % hi
M536x504S38800464x496 % .

M518x518S10043488x483S20500482x507 % me
M507x511S14720493x489 % b
M511x510S19220490x491 % i
M512x515S1dc20488x485 % l
M512x515S1dc20488x485 % l
M537x504S38700463x496 % ,
M508x515S10e20493x485 % v
M511x510S19220490x491 % i
M509x510S16d20492x490 % c
M510x508S1f720490x493 % a
M508x515S11a20493x485 % r
M508x508S20320493x493 % s
M537x504S38700463x496 % ,
M536x511S38a00464x490 % :
M536x511S38a00464x490 % :
M536x504S38800464x496 % .

M544x531S20500509x515S10011523x501S20500520x488S2ff00482x483 % deaf
M518x518S10043488x483S20500482x507 % me
M536x504S38800464x496 % .

M528x560S22a03497x526S17107479x534S16d21473x540S17107512x511S2ff00482x483 % wife
M507x511S14720493x489 % b
M508x508S14a20493x493 % e
M512x515S1dc20488x485 % l
M511x510S19220490x491 % i
M511x513S11920490x487 % n
M508x515S10110492x485 % d
M510x508S1f720490x493 % a
M537x504S38700463x496 % ,
M515x519S10047485x498S26507501x481 % she
M544x531S20500509x515S10011523x501S20500520x488S2ff00482x483 % deaf
M536x504S38800464x496 % .

M511x516S14400489x485 % 4
M522x525S15a50478x475S22a04483x510S2d508500x511 % children
M537x504S38700463x496 % ,
M518x540S14408493x509S10003487x460S20500483x498S20500495x493S20500508x499S2d508489x483 % first three of four
M532x549S2ff00482x483S2ea08495x524S10012502x510 % hearing
M533x525S14408467x494S10003503x476S20500492x497 % last one of four
M544x531S20500509x515S10011523x501S20500520x488S2ff00482x483 % deaf
M518x523S30300482x477 % nod
M533x525S14408467x494S10003503x476S20500492x497 % last one of four
M537x504S38700463x496 % ,
M508x515S10e00493x485 % 2
M545x522S26500524x507S18510520x489S22104527x476S2ff00482x483 % boy
M518x540S14408493x509S10003487x460S20500483x498S20500508x499S2d508489x483 % first and third of four
M525x554S1f540510x509S22a03486x540S20e00497x531S2ff00482x483 % girl
M521x537S14408479x506S10003491x464S20500482x492S20500504x505S2d509495x489 % second and fourth of four
M537x504S38700463x496 % ,
M517x531S14408491x500S10003487x469S20500484x492 % first of four
M512x515S1dc20488x485 % l
M508x508S17620492x492 % o
M515x508S1f000486x493 % g
M510x508S1f720490x493 % a
M511x513S11920490x487 % n
M537x504S38700463x496 % ,
M518x534S14408482x503S10003488x467S20500485x490 % second of four
M515x515S14020486x485 % k
M508x508S14a20493x493 % e
M512x515S1dc20488x485 % l
M508x508S20320493x493 % s
M508x508S14a20493x493 % e
M514x510S19a20486x490 % y
M537x504S38700463x496 % ,
M522x532S14408478x501S10003492x468S20500488x491 % third of four
M507x511S14720493x489 % b
M508x508S14a20493x493 % e
M511x513S11920490x487 % n
M537x504S38700463x496 % ,
M529x527S14408471x496S10003499x474S20500493x496 % fourth of four
M508x508S20320493x493 % s
M510x508S1f720490x493 % a
M508x515S11a20493x485 % r
M510x508S1f720490x493 % a
M515x508S11502485x493 % h
M536x504S38800464x496 % .
\end{center}
\end{multicols}

\subsection{Story 2.B}

\begin{multicols}{5}
\begin{center}
M547x518S2ff00482x483S22a07534x476S14711519x487 % hi
M536x504S38800464x496 % .

M518x518S10043488x483S20500482x507 % me
M515x515S14020486x485 % k
M508x508S14a20493x493 % e
M512x515S1dc20488x485 % l
M508x508S20320493x493 % s
M508x508S14a20493x493 % e
M514x510S19a20486x490 % y
M537x504S38700463x496 % ,
M508x515S10e20493x485 % v
M511x510S19220490x491 % i
M509x510S16d20492x490 % c
M510x508S1f720490x493 % a
M508x515S11a20493x485 % r
M508x508S20320493x493 % s
M536x504S38800464x496 % .

M532x549S2ff00482x483S2ea08495x524S10012502x510 % hearing
M518x518S10043488x483S20500482x507 % me
M536x504S38800464x496 % .

M518x535S2ff00482x483S20500494x520S14c10471x504 % mother
M518x518S2ff00482x483S20500495x469S14c10468x453 % father
M544x531S20500509x515S10011523x501S20500520x488S2ff00482x483 % deaf
M536x504S38800464x496 % .

M518x518S2ff00482x483S20500495x469S14c10468x453 % father
M522x525S11541498x491S11549479x498S20600489x476 % name
M507x511S14720493x489 % b
M511x510S19220490x491 % i
M512x515S1dc20488x485 % l
M512x515S1dc20488x485 % l
M536x504S38800464x496 % .

M518x535S2ff00482x483S20500494x520S14c10471x504 % mother
M507x511S14720493x489 % b
M508x508S14a20493x493 % e
M512x515S1dc20488x485 % l
M511x510S19220490x491 % i
M511x513S11920490x487 % n
M508x515S10110492x485 % d
M510x508S1f720490x493 % a
M536x504S38800464x496 % .

M532x518S18049468x483S18041507x483S20500486x507S20500504x507 % have
M508x515S10e00493x485 % 2
M538x568S1dc51508x466S1dc4a490x544S1dc42464x526S20500475x553S22b03501x512S2ff00482x483 % brother
M508x515S10000493x485 % 1
M539x571S2ff00482x483S1dc50515x494S1dc0a500x547S1dc02474x530S20500487x557S22a03509x524 % siste
M536x504S38800464x496 % .

M517x531S14408491x500S10003487x469S20500484x492 % first of four
M538x568S1dc51508x466S1dc4a490x544S1dc42464x526S20500475x553S22b03501x512S2ff00482x483 % brother
M537x504S38700463x496 % ,
M512x515S1dc20488x485 % l
M508x508S17620492x492 % o
M515x508S1f000486x493 % g
M510x508S1f720490x493 % a
M511x513S11920490x487 % n
M537x504S38700463x496 % ,
M532x549S2ff00482x483S2ea08495x524S10012502x510 % hearing
M537x504S38700463x496 % ,
M528x528S22b03504x472S17107478x502S16d21472x508 % married
M537x504S38700463x496 % ,
M508x515S10e00493x485 % 2
M522x525S15a50478x475S22a04483x510S2d508500x511 % children
M536x504S38800464x496 % .

M518x534S14408482x503S10003488x467S20500485x490 % second of four
M518x518S10043488x483S20500482x507 % me
M537x504S38700463x496 % ,
M516x525S10000492x495S2e806484x475 % single
M536x504S38800464x496 % .

M522x532S14408478x501S10003492x468S20500488x491 % third of four
M538x568S1dc51508x466S1dc4a490x544S1dc42464x526S20500475x553S22b03501x512S2ff00482x483 % brother
M537x504S38700463x496 % ,
M507x511S14720493x489 % b
M508x508S14a20493x493 % e
M511x513S11920490x487 % n
M537x504S38700463x496 % ,
M535x531S10140504x469S10148484x469S20500498x473S28905508x504S2891d466x504S2fb04494x525 % divorced
M537x504S38700463x496 % ,
M508x521S17610492x505S26500494x479 % 0
M522x525S15a50478x475S22a04483x510S2d508500x511 % children
M536x504S38800464x496

M529x527S14408471x496S10003499x474S20500493x496 % four of four
M539x571S2ff00482x483S1dc50515x494S1dc0a500x547S1dc02474x530S20500487x557S22a03509x524 % sister
M537x504S38700463x496 % ,
M508x508S20320493x493 % s
M510x508S1f720490x493 % a
M508x515S11a20493x485 % r
M510x508S1f720490x493 % a
M515x508S11502485x493 % h
M537x504S38700463x496 % ,
M544x531S20500509x515S10011523x501S20500520x488S2ff00482x483 % deaf
M537x504S38700463x496 % ,
M516x525S10000492x495S2e806484x475 % single
M536x504S38800464x496 % .
\end{center}
\end{multicols}

\end{document}

