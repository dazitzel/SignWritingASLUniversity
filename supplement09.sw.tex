\documentclass{article}

\usepackage{fontspec}
\usepackage{fullpage}
\usepackage{multicol}
\usepackage{multirow}
\usepackage{tikz}

\begin{document}

\newfontfamily\swfill{SuttonSignWritingFill.ttf}
\newfontfamily\swline{SuttonSignWritingLine.ttf}
\newcommand{\bul}{\hfil$\bullet$&}
\renewenvironment{glossary}{\begin{multicols}{5}\begin{center}}{\end{center}\end{multicols}}
\setcounter{secnumdepth}{0}
\setlength{\columnseprule}{1pt}

\section{Supplement For Lesson 9}

\subsection{Objectives}

\begin{tabular}{p{1cm}p{14cm}}
\bul I have completed the objectives for this lesson.\\
\bul I know which base symbols are in Symbol Group hit wall.\\
\bul I know which base symbols are in Symbol Group hit floor.\\
\bul I am able to write and sign the Cup, Thumb Side handshape.\\
\bul I am able to write and sign the Cup, No Thumb handshape.\\
\bul I am able to write and sign the Circle handshape.\\
\bul I am able to write and sign the Hinge handshape.\\
\bul I am able to write and sign the Hinge, Thumb Side handshape.\\
\bul I am able to write and sign the Hinge, No Thumb handshape.\\
\bul I am able to write and sign the Angle handshape.\\
\bul I am able to draw and sign the hinge finger movement.\\
\bul I am able to draw the gradual mark.\\
\bul I am able to draw ears.\\
\bul I am able to recognize the vocabulary for this lesson.\\
\bul I am able to read the practice sentences for this lesson.\\
\bul I am able to read the practice story for this lesson.\\
\end{tabular}

\subsection{Symbol Group Hit Wall}

The seventeenth Symbol Group we informally call hit wall.
Its official name is ``Curves Hit Wall Plane'', but you can now list off the symbol groups as ``one, two, three, four, five, six, seven, eight, nine, thumb, contact, fingers, wall, diagonal, floor, curve wall, hit wall''.

\begin{center}
\begin{tabular}{rcrc}
\textbf{Base Symbol}&\textbf{Example}&\textbf{Base Symbol}&\textbf{Example}\\
Curve Hits Front Wall               &B509x512S2a600492x489&Hump Hits Front Wall           &B509x518S2a700491x483\\
Loop Hits Front Wall                &B512x524S2a800488x477&Wave Hits Front Wall           &B509x516S2a900492x485\\
Rotation Single Hits Front Wall     &B514x512S2aa00486x489&Rotation Double Hits Front Wall&B518x512S2ab00482x489\\
Rotation Alternating Hits Front Wall&B520x513S2ac00480x488&Curve Hits Chest               &B507x512S2ad00494x489\\
Hump Hits Chest                     &B507x518S2ae00494x482&Loop Hits Chest                &B510x524S2af00490x477\\
Wave Hits Chest                     &B510x518S2b000491x482&Rotation Single Hits Chest     &B514x512S2b100486x489\\
Rotation Double Hits Chest          &B518x512S2b200482x489&Rotation Alternating Hits Chest&B520x513S2b300480x488\\
Wave Diagonal Path Small            &B511x519S2b400489x482&Wave Diagonal Path Medium      &B513x526S2b500487x475\\
Wave Diagonal Path Large            &B519x526S2b600481x475\\
\end{tabular}
\end{center}

\subsection{Symbol Group Hit Floor}

The eighteenth Symbol Group we informally call hit floor.
\begin{tabular}{p{1cm}p{14cm}}
\bul I know which base symbols are in Symbol Group hit floor.\\
\bul I am able to write and sign the Cup, Thumb Side handshape.\\
\bul I am able to write and sign the Cup, No Thumb handshape.\\
\bul I am able to write and sign the Circle handshape.\\
\bul I am able to write and sign the Hinge handshape.\\
\bul I am able to write and sign the Hinge, Thumb Side handshape.\\
\bul I am able to write and sign the Hinge, No Thumb handshape.\\
\bul I am able to write and sign the Angle handshape.\\
\bul I am able to draw and sign the hinge finger movement.\\
\bul I am able to draw the gradual mark.\\
\bul I am able to draw ears.\\
\bul I am able to recognize the vocabulary for this lesson.\\
\bul I am able to read the practice sentences for this lesson.\\
\bul I am able to read the practice story for this lesson.\\
\end{tabular}
\begin{center}\textbf{\Huge This is where we are!}\end{center}\end{document}
