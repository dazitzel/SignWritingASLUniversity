\documentclass{article}

\usepackage{fontspec}
\usepackage{fullpage}
\usepackage{multicol}
\usepackage{multirow}
\usepackage{tikz}

\begin{document}

\newfontfamily\swfill{SuttonSignWritingFill.ttf}
\newfontfamily\swline{SuttonSignWritingLine.ttf}
\newcommand{\bul}{\hfil$\bullet$&}
\renewenvironment{glossary}{\begin{multicols}{5}\begin{center}}{\end{center}\end{multicols}}
\setcounter{secnumdepth}{0}
\setlength{\columnseprule}{1pt}

\section{Supplement For Lesson 15}

\subsection{Objectives}

\begin{tabular}{p{1cm}p{14cm}}
\bul I have completed the objectives for this lesson.\\
\bul I know which base symbols are in Symbol Group detail.\\
\bul I know which base symbols are in Symbol Group punctuation.\\
\bul I am able to write and sign the Thumb handshape.\\
\bul I am able to write and sign the Thumb Heel handshape.\\
\bul I am able to write and sign the Thumb Side Diagonal handshape.\\
\bul I am able to write and sign the Thumb Side Bent handshape.\\
\bul I am able to write and sign the Thumb Between Index Middle handshape.\\
\bul I am able to write and sign the Fist handshape.\\
\bul I am able to write and sign the Fist Heel handshape.\\
\bul I am able to recognize the vocabulary for this lesson.\\
\bul I am able to read the practice sentences for this lesson.\\
\bul I am able to read the practice story for this lesson.\\
\end{tabular}

\subsection{Symbol Group Detail}

The twenty-ninth Symbol Group we informally call detail.
In conversational ASL, you should never need these but if you want to add a note to yourself about how to get the correct accent these are the symbols to you.
Its official name is ``Detailed Location'', but you can now list off the symbol groups as ``one, two, three, four, five, six, seven, eight, nine, thumb, contact, fingers, wall, diagonal, floor, curve wall, hit wall, hit floor, curve floor, circles, timing, head, eyes, middle, lips, mouth, trunk, limb, detail''.

\begin{center}
\begin{tabular}{rcrc}
\textbf{Base Symbol}&\textbf{Example}&\textbf{Base Symbol}&\textbf{Example}\\
Location Space Wall Plane&B521x521S37f00480x480&Location Space Floor Plane&B521x521S38000480x480\\
Location Height          &B522x520S38100479x481&Location Width            &B514x522S38200486x479\\
Location Depth           &B522x511S38300479x490&Location Head Neck        &B518x521S38400482x480\\
Location Torso           &B514x524S38500487x476&Location Limbs Digits     &B514x519S38600487x482\\
\end{tabular}
\end{center}

\subsection{Symbol Group Punctuation}

The thirtieth, and final, Symbol Group we informally call punctuation.
Its official name is also ``Punctuation'', but you can now list off the symbol groups as ``one, two, three, four, five, six, seven, eight, nine, thumb, contact, fingers, wall, diagonal, floor, curve wall, hit wall, hit floor, curve floor, circles, timing, head, eyes, middle, lips, mouth, trunk, limbs, detail, punctuation''.

\begin{center}
\begin{tabular}{rcrc}
\begin{tabular}{p{1cm}p{14cm}}
\bul I know which base symbols are in Symbol Group punctuation.\\
\bul I am able to write and sign the Thumb handshape.\\
\bul I am able to write and sign the Thumb Heel handshape.\\
\bul I am able to write and sign the Thumb Side Diagonal handshape.\\
\bul I am able to write and sign the Thumb Side Bent handshape.\\
\bul I am able to write and sign the Thumb Between Index Middle handshape.\\
\bul I am able to write and sign the Fist handshape.\\
\bul I am able to write and sign the Fist Heel handshape.\\
\bul I am able to recognize the vocabulary for this lesson.\\
\bul I am able to read the practice sentences for this lesson.\\
\bul I am able to read the practice story for this lesson.\\
\end{tabular}
\end{tabular}\end{center}\begin{center}\textbf{\Huge This is where we are!}\end{center}\end{document}
