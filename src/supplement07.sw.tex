\documentclass{article}

\usepackage{fontspec}
\usepackage{fullpage}
\usepackage{multicol}
\usepackage{multirow}
\usepackage{tikz}

\begin{document}

\newfontfamily\swfill{SuttonSignWritingFill.ttf}
\newfontfamily\swline{SuttonSignWritingLine.ttf}
\newcommand{\bul}{\hfil$\bullet$&}
\renewenvironment{glossary}{\begin{multicols}{5}\begin{center}}{\end{center}\end{multicols}}
\setcounter{secnumdepth}{0}
\setlength{\columnseprule}{1pt}

\section{Supplement For Lesson 7}

\begin{center}
\it
Objectives inspired by, vocabulary transcribed from, and sentences and story by Bill Vicars.

Handshape photos by Adam Frost.

No endorsement implied nor given by either.
\end{center}

\subsection{Objectives}

\begin{tabular}{p{1cm}p{14cm}}
\bul I have completed the objectives for this lesson.\\
\bul I understand how dominant hand affects SignWriting.\\
\bul I am able to read the numbers 1,000--999,999.\\
\bul I understand how the ``ABCOS15'' handshapes fit into ASL SignWriting.\\
\bul I am able to demonstrate the meaning and form of the symbol groups in the detail category.\\
\bul I know which base symbols are in Symbol Groups wall and diagonal.\\
\bul I am able to draw the fist heel palmshape in all forms.\\
\bul I am able to draw and demonstrate what fill six means.\\
\bul I am able to read, write, and sign one third of the ASL handshapes in symbol group five.\\
\bul I am able to recognize the vocabulary for this lesson.\\
\bul I am able to read the practice sentences for this lesson.\\
\bul I am able to read the practice story for this lesson.\\
\end{tabular}

\subsection{Dominant Hand and SignWriting}

We covered this long ago back in lesson one.
The full answer is that dominant hand can either have no effect on SignWriting or completely mirror all the handshapes and directions of each word.

If your dominant hand happens to be right, then the sign for clean happens to be B528x522S15a37473x486S15a51482x499S26507515x479S21100500x492 and that is the way we consistently write it.
But the sign for clean is also B517x529S15a3f494x503S15a59493x506S26511484x472S21100498x487 and when reading something written by someone else we must also accept this spelling.
Why?
As you become more comfortable with ASL and SignWriting you will internalize it as being in your voice --- you will process speech/signing as if you are the one speaking/signing.
So if someone has written you a note and it happens to be left hand dominant, that just means that it is how that person thinks.
Normally you might expect a disclaimer along the lines of ``pending teacher approval'' but in this case I'm going to say that if your teacher does not accept it then they are wrong --- it must be acceptable.

Which brings up the question as to why these supplements don't include both spellings?
Primarily it is because of space --- twice as many pages for sample sentences and stories and also the fear about overwhelming the reader.
A secondary reason is that it is well established that right hand dominant is the ``standard'' spelling --- if you find a book in ASL it will be right hand dominant.

So what effect does dominant hand have on SignWriting, usually none.
In cases where it does, here is the effect.

\begin{tabular}{p{1cm}p{14cm}}
\bul Each movement arrow is mirrored.\\
\bul Many movement arrows have their fill changed to reflect the change in handedness.\\
\bul Each hand is mirrored.\\
\bul Each symbol is moved to reverse the horizontal spacing --- vertical spacing remains the same.\\
\bul Noses and other items that we only have one of remain unchanged.\\
\end{tabular}

\subsection{The Numbers 1,000 Through 999,999}

In SignWriting you will almost exclusively be written as a vertical list of the digits for zero through nine.
This is just like English using digits to write out large numbers.
This lesson will, in contrast to normal usage, be showing the full words --- instead of 1,234 you will see one thousand two hundred thirty four.
Just like in English, writing the full words takes more space but given the goal of this supplement it is required.

In future lessons we will eventually switch to using digits, as you would normally see in SignWriting, but you must remember to interpret the digits and use the correct signs when speaking in ASL.

\begin{center}
\begin{tabular}{*{5}{c}}
\textbf{1,000}&\textbf{2,000}&\textbf{3,000}&\textbf{4,000}&\textbf{5,000}\\
B521x540S20500479x525S15a38492x513S18051496x515S10020502x461S22a04503x495&
B521x540S20500479x525S15a38492x513S18051496x515S10e20501x461S22a04503x495&
B521x540S20500479x525S15a38492x513S18051496x515S22a04503x495S11e20494x461&
B522x540S20500479x525S15a38492x513S18051496x515S22a04503x495S14420500x461&
B521x540S20500479x525S15a38492x513S18051496x515S22a04503x495S14c20498x461\\
\textbf{6,000}&\textbf{7,000}&\textbf{8,000}&\textbf{9,000}\\
B521x540S20500479x525S15a38492x513S18051496x515S22a04503x495S18720500x461&
B522x539S20500479x524S15a38492x512S18051496x514S22a04503x494S1a520501x462&
B522x539S20500479x524S15a38492x512S18051496x514S22a04503x494S1bb20501x462&
B523x540S20500478x525S15a38491x513S18051495x515S22a04502x495S1ce20501x461\\
\end{tabular}
\end{center}

\subsection{The ``ABCOS15'' Handshapes and ASL SignWriting}

The simple takeaway from ``ABCOS15'' is that SignWriting has B510x508S1f708490x493, B506x514S15a08494x487, B509x510S16d08492x490, B508x508S17600492x492, B508x508S20308493x493, B508x515S10008493x485, and B512x516S14c08489x485;
but there is so much more to consider.

English, at least American English, now has words and phrases which were originally borrowed from other languages like \textsc{rsvp}, je ne sais quoi, adios, lariat.
As these words and phrases become more anglicized, the pronunciation and even the meaning adjusted.
\textsc{Rsvp} is a literal request for a response regardless of the response, but for most speakers of American English it means respond but only if you are coming.
\emph{La reata} literally means ``the rope'' but was merged into a single word, had the last syllable dropped, and the stress moved forward, as well as becoming a special form of its original meaning.

English is not the only language to do these types of things.
In fact, this is so common it would be rather strange for ASL to \emph{not} do this.
The difference is that while ASL \emph{can} borrow from other sign languages, it is \emph{more likely} to borrow from English spelling and/or pantomiming in spaces not occupied by current words.
This is how we start with and still have initialized words in ASL as well as words that are visually descriptive, but as native speakers become more comfortable with the concept and make it their own the handshapes will move from ``well formed letters'' into some basic handshapes which, movements, and locations.
That is, words that start with B508x515S11500493x485 or B510x513S18d00490x488, are likely to become B512x508S18200489x493 over time;
many words that start with both hands moving are likely to simplify to either a dominant hand moving while the non-dominant hand remains stationary or both hands making the same movement;
and words that start around the waist are likely to move up over time.

For the dominant hand, the number of handshapes available is rather larger --- these lessons cover what has been identified as the basic handshapes for the dominant hand and it's just over eighty.
For the non-dominant hand, these handshapes are likely to simplify to this much smaller set of about seven shapes both because these are the standard shapes and because native speakers learning new words aren't interested in the exact shape as they learn to speak but in maintaining correct understanding.

\subsection{The Detail Category}

We informally call this category detail, though it's official name is ``Detailed Location'' and it has one base symbol in it with the same name.

\begin{center}
\begin{tabular}{ccc}
\textbf{Symbol}\\
\textbf{Group}&\textbf{Name}&\textbf{Example}\\
\textbf{29}&Detail&B521x521S37f00480x480\\
\end{tabular}
\end{center}

\subsection{The Symbol Groups Wall and Diagonal}

The thirteenth Symbol Group we informally call wall, though it's official name is ``Straight Wall Plane''.
Symbol Group Straight Wall Plane (Wall) has all types of vertical movement --- up and down though left and right movement can also be shown.
Each of these arrows have a pair of tails, to remind us of a rocket ship blasting off vertically.

\begin{center}
\begin{tabular}{rcrc}
\textbf{Base Symbol}&\textbf{Example}&\textbf{Base Symbol}&\textbf{Example}\\
Single Straight Movement, Wall Plane Small&B507x508S22a00494x493&Single Straight Movement, Wall Plane Medium &B508x515S22b00492x485\\
Single Straight Movement, Wall Plane Large&B508x521S22c00492x479&Single Straight Movement, Wall Plane Largest&B508x525S22d00492x475\\
Single Wrist Flex, Wall Plane             &B509x509S22e00492x491&Double Straight Movement, Wall Plane        &B513x507S22f00488x493\\
Double Wrist Flex, Wall Plane             &B513x509S23000488x491&Double Alternating Movement, Wall Plane     &B513x509S23100487x492\\
Double Alternating Wrist Flex, Wall Plane &B513x510S23200487x490&Cross Movement, Wall Plane                  &B515x513S23300485x487\\
Triple Straight Movement, Wall Plane      &B519x507S23400482x493&Triple Wrist Flex, Wall Plane               &B519x509S23500482x491\\
Triple Alternating Movement, Wall Plane   &B520x509S23600481x492&Triple Alternating Wrist Flex, Wall Plane   &B520x511S23700481x490\\
Bend, Wall Plane Small                    &B509x513S23800492x488&Bend, Wall Plane Medium                     &B510x515S23900490x485\\
Bend, Wall Plane Large                    &B513x521S23a00487x479&Corner, Wall Plane Small                    &B510x511S23b00491x490\\
Corner, Wall Plane Medium                 &B512x514S23c00489x487&Corner, Wall Plane Large                    &B514x517S23d00487x483\\
Corner, Wall Plane with Rotation          &B515x516S23e00485x484&Check, Wall Plane Small                     &B511x514S23f00490x486\\
Check, Wall Plane Medium                  &B513x517S24000487x483&Check, Wall Plane Large                     &B514x520S24100486x481\\
Box, Wall Plane Small                     &B512x511S24200488x489&Box, Wall Plane Medium                      &B515x514S24300486x487\\
Box, Wall Plane Large                     &B517x517S24400484x484&Zigzag, Wall Plane Small                    &B509x516S24500491x485\\
Zigzag, Wall Plane Medium                 &B512x520S24600488x481&Zigzag, Wall Plane Large                    &B513x522S24700487x479\\
Peaks, Wall Plane Small                   &B508x514S24800492x486&Peaks, Wall Plane Medium                    &B510x518S24900491x482\\
Peaks, Wall Plane Large                   &B511x524S24a00490x477&Travel Rotation, Single Wall Plane          &B511x513S24b00489x487\\
Travel Rotation, Double Wall Plane        &B511x518S24c00489x483&Travel Rotation, Alternating Wall Plane     &B512x518S24d00488x483\\
Travel Rotation, Single Floor Plane       &B511x515S24e00490x485&Travel Rotation, Double Floor Plane         &B510x518S24f00490x482\\
Travel Rotation, Alternating Floor Plane  &B511x518S25000489x482&Travel Shaking, Wall Plane                  &B509x517S25100491x484\\
Travel Arm Spiral, Wall Plane Single      &B513x518S25200488x483&Travel Arm Spiral, Wall Plane Double        &B513x521S25300488x479\\
Travel Arm Spiral, Wall Plane Triple      &B513x525S25400488x475\\
\end{tabular}
\end{center}

The fourteenth Symbol Group we informally call diagonal, though it's official name is ``Straight Diagonal Movement''.
Symbol Group Straight Diagonal Movement (Diagonal) has all ways the movement that is both horizontal and vertical at the same time.
It is shown to be primarily vertical but the movement away from you will have an extra line --- like looking at the tail of a jet moving away from you.
Movement towards you will have an extra circle --- like watching the nose of a jet coming straight at you.

\begin{center}
\begin{tabular}{rcrc}
\textbf{Base Symbol}&\textbf{Example}&\textbf{Base Symbol}&\textbf{Example}\\
Diagonal Away Movement Small   &B507x511S25500494x489&Diagonal Away Movement Medium    &B508x515S25600492x485\\
Diagonal Away Movement Large   &B508x521S25700492x479&Diagonal Away Movement Largest   &B508x525S25800492x475\\
Diagonal Towards Movement Small&B507x512S25900493x489&Diagonal Towards Movement Medium &B508x515S25a00492x485\\
Diagonal Towards Movement Large&B508x521S25b00492x479&Diagonal Towards Movement Largest&B508x525S25c00492x475\\
Diagonal Between Away Small    &B507x510S25d00493x490&Diagonal Between Away Medium     &B508x515S25e00492x485\\
Diagonal Between Away Large    &B508x520S25f00492x480&Diagonal Between Away Largest    &B508x525S26000492x475\\
Diagonal Between Towards Small &B507x510S26100493x491&Diagonal Between Towards Medium  &B508x515S26200492x485\\
Diagonal Between Towards Large &B508x521S26300492x479&Diagonal Between Towards Largest &B508x525S26400492x475\\
\end{tabular}
\end{center}

Before you can consider this lesson complete, you need to be able to list off the symbol groups as:
``one, two, three, four, five, six, seven, eight, nine, thumb;''
``contact, finger, wall, diagonal.''

Some additional help when remembering the second set of base symbols.
Contact and finger you will just have to remember, but now point to the wall and rotate your arm till your index finger points down to the floor.

This is the order followed by the base symbols in the second set --- wall first, floor last.
You can consider this lesson complete when you can remember ``contact, finger, wall, diagonal,'' but you should be able to guess that the next symbol is floor because ``wall first, floor last.''

\subsection{The Fist Heel Palmshape}

The Fist Heel Palmshape only comes as fill 2 which in the case of the primary rotation means palm up.
The idea is that when you are focused on the heel, you can see both the palm and the back of the hand at the same time so fills 1, 3, 4, and 6 cannot occur.
The reason fill 5 is missing is because of the physiology of human arms --- you won't be using the heel handshape with palm towards the signer.

\begin{center}
\begin{tabular}{r*{6}{c}}
&\textbf{Fill 1}&\textbf{Fill 2}&\textbf{Fill 3}&\textbf{Fill 4}&\textbf{Fill 5}&\textbf{Fill 6}\\
\textbf{Right}&---&B508x506S20410493x495&---&---&---&---\\
\textbf{Left} &---&B508x506S20418493x495&---&---&---&---\\
\end{tabular}
\end{center}

\subsection{The Sixth Fill}

\subsubsection{Hand Symbols}

\begin{center}
B508x515S10050493x485 B508x515S10e50493x485 B512x515S11e50489x485
\end{center}

Any handshape symbol drawn in the sixth fill means that the signer's palm is facing down.
For all the hand symbols, the empty portion represents the signer's palm and the filled portion represents the back of the hand.
So for fill six the palm is completely filled in.

\subsubsection{Everything Else}

\begin{center}
B518x518S30a50482x483
\end{center}

The fills for other categories tend to be a bit more variable.
Here we have the left eyebrow raised by itself.

\subsection{First ASL Handshapes From Symbol Group Five}

The twenty one handshapes in Symbol Group Five used by ASL in order are:
{\it
Five Fingers Spread;
Five Fingers Spread Heel;
Five Fingers Spread, Four Bent;
Five Fingers Spread, Four Bent Heel;
Five Fingers Spread Bent;
Five Fingers Spread Bent Heel;
Five Fingers Spread Cup;
}
Five Fingers Spread Cup Open;
Five Fingers Spread Hinge;
Flat Hand;
Flat Heel;
Flat, Thumb Side;
Flat, Thumb Side Heel;
Cup;
Cup, Thumb Side;
Cup, No Thumb;
Circle;
Hinge;
Hinge, Thumb Side
Hinge, No Thumb
and Angle.

\subsubsection{The Five Fingers Spread Handshape}

\begin{center}
\begin{tabular}{r*{6}{c}}
&\textbf{Fill 1}&\textbf{Fill 2}&\textbf{Fill 3}&\textbf{Fill 4}&\textbf{Fill 5}&\textbf{Fill 6}\\
\multirow{2}{*}{\textbf{Right}}&
B512x516S14c00489x485&
B512x516S14c10489x485&
B512x516S14c20489x485&
B512x516S14c30489x485&
B512x516S14c40489x485&
B512x516S14c50489x485\\
&
\includegraphics[scale=0.1]{images/05-01-1.jpg}&
\includegraphics[scale=0.1]{images/05-01-2.jpg}&
\includegraphics[scale=0.1]{images/05-01-3.jpg}&
\includegraphics[scale=0.1]{images/05-01-4.jpg}&
\includegraphics[scale=0.1]{images/05-01-5.jpg}&
\includegraphics[scale=0.1]{images/05-01-6.jpg}\\
\textbf{Left}&
B512x516S14c08489x485&
B512x516S14c18489x485&
B512x516S14c28489x485&
B512x516S14c38489x485&
B512x516S14c48489x485&
B512x516S14c58489x485\\
\end{tabular}
\end{center}

\subsubsection{The Five Fingers Spread Heel Handshape}

\begin{center}
\begin{tabular}{r*{6}{c}}
&\textbf{Fill 1}&\textbf{Fill 2}&\textbf{Fill 3}&\textbf{Fill 4}&\textbf{Fill 5}&\textbf{Fill 6}\\
\multirow{2}{*}{\textbf{Right}}&
\multirow{2}{*}{---}&
B515x509S14d10485x491&
\multirow{2}{*}{---}&
\multirow{2}{*}{---}&
\multirow{2}{*}{---}&
\multirow{2}{*}{---}\\
&
&
\includegraphics[scale=0.1]{images/05-02-2.jpg}\\
\textbf{Left}&
---&
B515x509S14d18485x491&
---&
---&
---&
---\\
\end{tabular}
\end{center}

\subsubsection{The Five Fingers Spread, Four Bent Handshape}

\begin{center}
\begin{tabular}{r*{6}{c}}
&\textbf{Fill 1}&\textbf{Fill 2}&\textbf{Fill 3}&\textbf{Fill 4}&\textbf{Fill 5}&\textbf{Fill 6}\\
\multirow{2}{*}{\textbf{Right}}&
B513x516S14e00488x485&
B513x516S14e10488x485&
B513x516S14e20488x485&
B513x516S14e30488x485&
B513x516S14e40488x485&
B513x516S14e50488x485\\
&
\includegraphics[scale=0.1]{images/05-03-1.jpg}&
\includegraphics[scale=0.1]{images/05-03-2.jpg}&
\includegraphics[scale=0.1]{images/05-03-3.jpg}&
\includegraphics[scale=0.1]{images/05-03-4.jpg}&
\includegraphics[scale=0.1]{images/05-03-5.jpg}&
\includegraphics[scale=0.1]{images/05-03-6.jpg}\\
\textbf{Left}&
B513x516S14e08488x485&
B513x516S14e18488x485&
B513x516S14e28488x485&
B513x516S14e38488x485&
B513x516S14e48488x485&
B513x516S14e58488x485\\
\end{tabular}
\end{center}

\subsubsection{The Five Fingers Spread, Four Bent Heel Handshape}

\begin{center}
\begin{tabular}{r*{6}{c}}
&\textbf{Fill 1}&\textbf{Fill 2}&\textbf{Fill 3}&\textbf{Fill 4}&\textbf{Fill 5}&\textbf{Fill 6}\\
\multirow{2}{*}{\textbf{Right}}&
\multirow{2}{*}{---}&
B515x508S14f10485x492&
\multirow{2}{*}{---}&
\multirow{2}{*}{---}&
\multirow{2}{*}{---}&
\multirow{2}{*}{---}\\
&
&
\includegraphics[scale=0.1]{images/05-04-2.jpg}\\
\textbf{Left}&
---&
B515x508S14f18485x492&
---&
---&
---&
---\\
\end{tabular}
\end{center}

\subsubsection{The Five Fingers Spread Bent Handshape}

\begin{center}
\begin{tabular}{r*{6}{c}}
&\textbf{Fill 1}&\textbf{Fill 2}&\textbf{Fill 3}&\textbf{Fill 4}&\textbf{Fill 5}&\textbf{Fill 6}\\
\multirow{2}{*}{\textbf{Right}}&
B513x516S15000488x485&
B513x516S15010488x485&
B513x516S15020488x485&
B513x516S15030488x485&
B513x516S15040488x485&
B513x516S15050488x485\\
&
\includegraphics[scale=0.1]{images/05-05-1.jpg}&
\includegraphics[scale=0.1]{images/05-05-2.jpg}&
\includegraphics[scale=0.1]{images/05-05-3.jpg}&
\includegraphics[scale=0.1]{images/05-05-4.jpg}&
\includegraphics[scale=0.1]{images/05-05-5.jpg}&
\includegraphics[scale=0.1]{images/05-05-6.jpg}\\
\textbf{Left}&
B513x516S15008488x485&
B513x516S15018488x485&
B513x516S15028488x485&
B513x516S15038488x485&
B513x516S15048488x485&
B513x516S15058488x485\\
\end{tabular}
\end{center}

\subsubsection{The Five Fingers Spread Bent Heel Handshape}

\begin{center}
\begin{tabular}{r*{6}{c}}
&\textbf{Fill 1}&\textbf{Fill 2}&\textbf{Fill 3}&\textbf{Fill 4}&\textbf{Fill 5}&\textbf{Fill 6}\\
\multirow{2}{*}{\textbf{Right}}&
\multirow{2}{*}{---}&
B515x508S15110486x492&
\multirow{2}{*}{---}&
\multirow{2}{*}{---}&
\multirow{2}{*}{---}&
\multirow{2}{*}{---}\\
&
&
\includegraphics[scale=0.1]{images/05-06-2.jpg}\\
\textbf{Left}&
---&
B515x508S15118486x492&
---&
---&
---&
---\\
\end{tabular}
\end{center}

\subsubsection{The Five Fingers Spread Cup Handshape}

\begin{center}
\begin{tabular}{r*{6}{c}}
&\textbf{Fill 1}&\textbf{Fill 2}&\textbf{Fill 3}&\textbf{Fill 4}&\textbf{Fill 5}&\textbf{Fill 6}\\
\multirow{2}{*}{\textbf{Right}}&
B508x513S15300493x488&
B508x513S15310493x488&
B508x513S15320493x488&
B508x513S15330493x488&
B508x513S15340493x488&
B508x513S15350493x488\\
&
\includegraphics[scale=0.1]{images/05-07-1.jpg}&
\includegraphics[scale=0.1]{images/05-07-2.jpg}&
\includegraphics[scale=0.1]{images/05-07-3.jpg}&
\includegraphics[scale=0.1]{images/05-07-4.jpg}&
\includegraphics[scale=0.1]{images/05-07-5.jpg}&
\includegraphics[scale=0.1]{images/05-07-6.jpg}\\
\textbf{Left}&
B508x513S15308493x488&
B508x513S15318493x488&
B508x513S15328493x488&
B508x513S15338493x488&
B508x513S15348493x488&
B508x513S15358493x488\\
\end{tabular}
\end{center}

\subsection{Vocabulary}

\begin{glossary}

\textbf{1,000}\\
AS10020S22a04S18051S15a38S20500M521x540S20500479x525S15a38492x513S18051496x515S10020502x461S22a04503x495

\textbf{2,000}\\
AS10e20S22a04S18051S15a38S20500M521x540S20500479x525S15a38492x513S18051496x515S10e20501x461S22a04503x495

\textbf{3,000}\\
AS11e20S22a04S18051S15a38S20500M521x540S20500479x525S15a38492x513S18051496x515S22a04503x495S11e20494x461

\textbf{4,000}\\
AS14420S22a04S18051S15a38S20500M522x540S20500479x525S15a38492x513S18051496x515S22a04503x495S14420500x461

\textbf{5,000}\\
AS14c20S22a04S18051S15a38S20500M521x540S20500479x525S15a38492x513S18051496x515S22a04503x495S14c20498x461

\textbf{6,000}\\
AS18720S22a04S20500S18051S15a38M521x540S20500479x525S15a38492x513S18051496x515S22a04503x495S18720500x461

\textbf{7,000}\\
AS1a520S22a04S18051S15a38S20500M522x539S20500479x524S15a38492x512S18051496x514S22a04503x494S1a520501x462

\textbf{8,000}\\
AS1bb20S22a04S18051S15a38S20500M522x539S20500479x524S15a38492x512S18051496x514S22a04503x494S1bb20501x462

\textbf{9,000}\\
AS1ce20S22a04S18051S15a38S20500M523x540S20500478x525S15a38491x513S18051495x515S22a04502x495S1ce20501x461

\textbf{and}\\
AS14c02S26506S18500M543x512S14c02458x489S26506496x492S18500519x494

\textbf{apple}\\
AS10612S2c400S2ff00M563x526S2ff00482x483S10612515x511S2c400523x483

\textbf{bologna}\\
AS17650S17658S22a06S22a22S2fb00S20350S20358M560x513S17650482x487S17654507x487S20350545x488S20350440x487S22a06526x489S22a12460x488S2fb04495x507

\textbf{candy}\\
AS10011S2e000S20500S2ff00M541x542S2ff00482x483S10011510x512S20500504x518S2e000520x485

\textbf{cereal}\\
AS10612S22a02S21800S2ff00M518x552S2ff00482x483S10612484x520S21800462x522S22a02494x539

\textbf{cheese}\\
AS14c20S14c0aS21100S2a400M524x546S2a400483x455S14c20484x500S14c0a493x522S21100477x532

\textbf{cookie}\\
AS15051S15a3aS20500S2df08S20500M523x537S15a3a490x494S20500482x481S20500513x481S2df08491x464S15051477x509

\textbf{cup}\\
AS16d40S22f04S15d39S20600M513x543S16d40492x457S22f04488x483S20600489x506S15d39487x520

\textbf{drink}\\
AS16d10S2d201S33b00M518x567S33b00482x483S16d10492x520S2d201490x544

\textbf{eat}\\
AS18507S20500S33b00M528x537S33b00482x483S18507502x510S20500518x509

\textbf{egg}\\
AS11501S11509S28905S2891dS20500S2fb04M540x541S20500496x459S11501506x479S11509471x479S28905513x516S2891d460x515S2fb04493x523

\textbf{food}\\
AS18507S20600S33b00M540x537S33b00482x483S18507502x510S20600518x509

\textbf{full}\\
AS15a51S20348S26603S20e00M529x519S15a51506x482S20348492x504S20e00489x489S26603472x481

\textbf{full}\\
AS14712S22b00S2ff00M518x570S2ff00482x483S14712488x519S22b00493x540

\textbf{hamburger}\\
AS16d21S17107S20800S16d29S1710fS20800M515x543S20800505x533S20800486x457S16d21488x471S1710f490x507S17107496x466S16d29492x513

\textbf{hotdog}\\
AS17650S17658S22a06S22a22S2fb00S20350S20358M560x513S17650482x487S17654507x487S20350545x488S20350440x487S22a06526x489S22a12460x488S2fb04495x507

\textbf{hungry}\\
AS15402S21100S22c04M513x542S15402488x458S21100493x478S22c04490x500

\textbf{kind (an in type)}\\
AS14240S14249S2e800S20500M524x532S14249491x508S14240477x480S2e800510x469S20500480x509

\textbf{milk}\\
AS16d40S21802M515x511S21802485x493S16d40497x490

\textbf{pizza}\\
AS11820S2470aS1f720M550x522S1f720530x506S11820451x478S2470a482x496

\textbf{popcorn}\\
AS1eb07S1eb0fS21d00S21d00S2eb00S2eb48S2fd04M524x546S1eb07504x465S1eb0f476x476S2eb48480x508S21d00478x465S2eb00507x499S21d00500x454S2fd04497x540

\textbf{sausage}\\
AS17650S17658S22a06S22a22S2fb00S20350S20358M560x513S17650482x487S17654507x487S20350545x488S20350440x487S22a06526x489S22a12460x488S2fb04495x507

\textbf{soup}\\
AS11502S15d09S2ea00S20f00M518x585S15d09463x556S11502481x570S2ea00494x521S20f00489x551S33b00482x483

\textbf{style}\\
AS14240S14249S2e800S20500M524x532S14249491x508S14240477x480S2e800510x469S20500480x509

\textbf{taste}\\
AS1c50fS20500S33b00M518x539S1c50f476x514S33b00482x483S20500500x519

\textbf{thousand}\\
AS18051S15a38S20500M521x514S20500479x499S15a38492x487S18051496x489

\textbf{type}
AS14240S14249S2e800S20500M524x532S14249491x508S14240477x480S2e800510x469S20500480x509

\textbf{water}\\
AS18610S20600S2ff00M534x549S20600512x513S2ff00482x483S18610492x519

\textbf{wish}\\
AS15402S21100S22c04M513x542S15402488x458S21100493x478S22c04490x500

\end{glossary}

\subsection{Practice Sheet 7.A}

\begin{multicols}{5}
\begin{center}

M508x515S10000493x485 % 1
M536x504S38800464x496 % .
M563x526S2ff00482x483S10612515x511S2c400523x483 % apple
M537x504S38700463x496 % ,
M516x526S1f000484x474S2e230497x494 % green
M537x504S38700463x496 % ,
M518x518S30a00482x483 % y/n
M510x523S10040495x493S26500491x478 % you
M516x540S1bb02488x461S14c02484x517S20e00499x502S26500499x483 % like
M528x537S33b00482x483S18507502x510S20500518x509 % eat
M536x507S38900464x493 % ?
\vfil
\columnbreak

M508x515S10e00493x485 % 2
M536x504S38800464x496 % .
M541x542S2ff00482x483S10011510x512S20500504x518S2e000520x485 % candy
M537x504S38700463x496 % ,
M510x523S10040495x493S26500491x478 % you
M516x540S1bb02488x461S14c02484x517S20e00499x502S26500499x483 % like
M518x518S30c00482x483 % \?
M524x532S14249491x508S14240477x480S2e800510x469S20500480x509 % kind
M536x507S38900464x493 % ?
\vfil
\columnbreak

M512x515S11e00489x485 % 3
M536x504S38800464x496 % .
M518x552S2ff00482x483S10612484x520S21800462x522S22a02494x539 % cereal
M537x504S38700463x496 % ,
M510x523S10040495x493S26500491x478 % you
M516x540S1bb02488x461S14c02484x517S20e00499x502S26500499x483 % like
M518x518S30c00482x483 % \?
M524x532S14249491x508S14240477x480S2e800510x469S20500480x509 % kind
M536x507S38900464x493 % ?
\vfil
\columnbreak

M511x516S14400489x485 % 4
M536x504S38800464x496 % .
M518x518S30a00482x483 % y/n
M510x523S10040495x493S26500491x478 % you
M516x540S1bb02488x461S14c02484x517S20e00499x502S26500499x483 % like
M523x537S15a3a490x494S20500482x481S20500513x481S2df08491x464S15051477x509 % cookie
M521x515S1f748479x500S1f740501x500S20500495x485 % with
M515x511S21802485x493S16d40497x490 % milk
M536x507S38900464x493 % ?
\vfil
\columnbreak

M512x516S14c00489x485 % 5
M536x504S38800464x496 % .
M534x549S20600512x513S2ff00482x483S18610492x519 % water
M510x523S10040495x493S26500491x478 % you
M518x567S33b00482x483S16d10492x520S2d201490x544 % drink
M541x520S26a00514x474S2ff00482x483S20e00521x490S1f710514x505 % daily
M537x504S38700463x496 % ,
M513x543S16d40492x457S22f04488x483S20600489x506S15d39487x520 % cup
M526x535S22a20494x501S14c08474x465S14c00503x465S20338478x520S20330508x520 % how many
M510x523S10040495x493S26500491x478 % you
M536x507S38900464x493 % ?
\vfil

\end{center}
\end{multicols}

\subsection{Practice Sheet 7.B}

\begin{multicols}{5}
\begin{center}

M509x515S18720491x486 % 6
M536x504S38800464x496 % .
M518x518S30c00482x483 % \?
M507x523S15a28494x496S26500493x477 % your
M540x543S1c507499x518S20600518x508S2ff00482x483 % favorite
M540x537S33b00482x483S18507502x510S20600518x509 % food
M553x518S2fb04492x512S26c0a538x483S26c12448x488S14c39468x483S14c31506x483 % what
M536x507S38900464x493 % ?
\vfil
\columnbreak

M511x514S1a520490x486 % 7
M536x504S38800464x496 % .
M516x526S1f000484x474S2e230497x494 % green
M540x541S20500496x459S11501506x479S11509471x479S28905513x516S2891d460x515S2fb04493x523 % egg
M543x512S14c02458x489S26506496x492S18500519x494 % and
M515x508S11502485x493 % h
M510x508S1f720490x493 % a
M510x513S18d20490x488 % m
M510x523S10040495x493S26500491x478 % you
M516x540S1bb02488x461S14c02484x517S20e00499x502S26500499x483 % like
M518x518S30a00482x483 % y/n
M510x523S10040495x493S26500491x478 % you
M536x507S38900464x493 % ?
\vfil
\columnbreak

M511x514S1bb20490x486 % 8
M536x504S38800464x496 % .
M518x553S21600497x545S16d10492x520S33b00482x483 % orange
M537x504S38700463x496 % ,
M518x518S30a00482x483 % y/n
M510x523S10040495x493S26500491x478 % you
M516x540S1bb02488x461S14c02484x517S20e00499x502S26500499x483 % like
M528x537S33b00482x483S18507502x510S20500518x509 % eat
M536x507S38900464x493 % ?
\vfil
\columnbreak

M511x515S1ce20489x485 % 9
M536x504S38800464x496 % .
M518x518S30a00482x483 % y/n
M510x523S10040495x493S26500491x478 % you
M518x570S2ff00482x483S14712488x519S22b00493x540 % full (under chin)
M536x507S38900464x493 % ?
\vfil
\columnbreak

M513x528S2a538494x472S1f540488x504 % 10
M536x504S38800464x496 % .
M518x518S30c00482x483 % \?
M510x523S10040495x493S26500491x478 % you
M540x543S1c507499x518S20600518x508S2ff00482x483 % favorite
M537x504S38700463x496 % ,
R515x543S20800505x533S20800486x457S16d21488x471S1710f490x507S17107496x466S16d29492x513 % hamburger
L560x513S17650482x487S17654507x487S20350545x488S20350440x487S22a06526x489S22a12460x488S2fb04495x507 % hotdog
M536x507S38900464x493 % ?
\vfil

\end{center}
\end{multicols}

\subsection{Practice Sheet 7.C}

\begin{multicols}{5}
\begin{center}

M512x520S10000489x490S21d00494x480 % 11
M536x504S38800464x496 % .
M518x518S30c00482x483 % \?
M513x542S15402488x458S21100493x478S22c04490x500 % hungry
M510x523S10040495x493S26500491x478 % you
M536x507S38900464x493 % ?
\vfil
\columnbreak

M509x521S10e00491x491S21d00491x480 % 12
M536x504S38800464x496 % .
M550x522S1f720530x506S11820451x478S2470a482x496 % pizza
M537x504S38700463x496 % ,
M518x518S30c00482x483 % \?
M510x523S10040495x493S26500491x478 % you
M516x540S1bb02488x461S14c02484x517S20e00499x502S26500499x483 % like
M524x532S14249491x508S14240477x480S2e800510x469S20500480x509 % kind
M536x507S38900464x493 % ?
\vfil
\columnbreak

M513x519S22114487x481S12d00489x489 % 13
M536x504S38800464x496 % .
M563x518S2ff00482x483S19200542x478S26a04519x487S20500504x501 % suppose
M510x523S10040495x493S26500491x478 % you
M525x526S10018476x477S10018497x496S2882a503x475 % go
M542x523S10058459x493S14c10482x477S20500476x501S2a502509x490 % movie
M537x504S38700463x496 % ,
M518x518S30a00482x483 % y/n
M510x523S10040495x493S26500491x478 % you
M516x540S1bb02488x461S14c02484x517S20e00499x502S26500499x483 % like
M528x537S33b00482x483S18507502x510S20500518x509 % eat
M524x546S1eb07504x465S1eb0f476x476S2eb48480x508S21d00478x465S2eb00507x499S21d00500x454S2fd04497x540 % popcorn
M536x507S38900464x493 % ?
\vfil
\columnbreak

M513x515S14700493x493S22114487x486 % 14
M536x504S38800464x496 % .
M518x585S15d09463x556S11502481x570S2ea00494x521S20f00489x551S33b00482x483 % soup
M537x504S38700463x496 % ,
M518x518S30c00482x483 % \?
M510x523S10040495x493S26500491x478 % you
M516x540S1bb02488x461S14c02484x517S20e00499x502S26500499x483 % like
M524x532S14249491x508S14240477x480S2e800510x469S20500480x509 % kind
M536x507S38900464x493 % ?
\vfil
\columnbreak

M513x518S22114487x483S15d00494x491 % 15
M536x504S38800464x496 % .
M563x526S2ff00482x483S10612515x511S2c400523x483 % apple
M537x504S38700463x496 % ,
M526x552S21600515x544S33b00482x483S20e00514x514S22a04513x528S10000487x511 % red
M537x504S38700463x496 % ,
M510x523S10040495x493S26500491x478 % you
M516x540S1bb02488x461S14c02484x517S20e00499x502S26500499x483 % like
M528x537S33b00482x483S18507502x510S20500518x509 % eat
M518x518S30a00482x483 % y/n
M510x523S10040495x493S26500491x478 % you
M536x507S38900464x493 % ?
\vfil

\end{center}
\end{multicols}

\subsection{Practice Sheet 7.D}

\begin{multicols}{5}
\begin{center}

M520x522S18700502x492S2e00e480x479 % 16
M536x504S38800464x496 % .
M523x537S15a3a490x494S20500482x481S20500513x481S2df08491x464S15051477x509 % cookie
M537x504S38700463x496 % ,
M518x518S30c00482x483 % \?
M510x523S10040495x493S26500491x478 % you
M516x540S1bb02488x461S14c02484x517S20e00499x502S26500499x483 % like
M524x532S14249491x508S14240477x480S2e800510x469S20500480x509 % kind
M536x507S38900464x493 % ?
\vfil
\columnbreak

M522x522S1a500501x494S2e00e478x478 % 17
M536x504S38800464x496 % .
M518x518S30c00482x483 % \?
M510x523S10040495x493S26500491x478 % you
M540x543S1c507499x518S20600518x508S2ff00482x483 % favorite
R563x526S2ff00482x483S10612515x511S2c400523x483 % apple
L518x553S21600497x545S16d10492x520S33b00482x483 % orange
M536x507S38900464x493 % ?
\vfil
\columnbreak

M523x522S1bb00502x492S2e00e478x479 % 18
M536x504S38800464x496 % .
M518x518S30a00482x483 % y/n
M507x523S15a28494x496S26500493x477 % your
M539x571S2ff00482x483S1dc50515x494S1dc0a500x547S1dc02474x530S20500487x557S22a03509x524 % sister
M516x540S1bb02488x461S14c02484x517S20e00499x502S26500499x483 % like
M540x541S20500496x459S11501506x479S11509471x479S28905513x516S2891d460x515S2fb04493x523 % egg
M536x507S38900464x493 % ?
\vfil
\columnbreak

M524x522S1ce00502x490S2e00e477x479 % 19
M536x504S38800464x496 % .
M563x518S2ff00482x483S19200542x478S26a04519x487S20500504x501 % suppose
M510x523S10040495x493S26500491x478 % you
M528x537S33b00482x483S18507502x510S20500518x509 % eat
M512x515S11e00489x485 % 3
M515x543S20800505x533S20800486x457S16d21488x471S1710f490x507S17107496x466S16d29492x513 % hamburger
M537x504S38700463x496 % ,
M518x518S30a00482x483 % y/n
M536x518S2b800511x454S15a10524x487S34700482x483 % will
M518x570S2ff00482x483S14712488x519S22b00493x540 % full (under chin)
M510x523S10040495x493S26500491x478 % you
M536x507S38900464x493 % ?
\vfil
\columnbreak

M517x513S22114484x488S1f420488x498 % 20
M536x504S38800464x496 % .
M518x518S30a00482x483 % y/n
M510x523S10040495x493S26500491x478 % you
M516x538S1bb02488x517S14c52484x463S2d200490x490 % don't like
M541x542S2ff00482x483S10011510x512S20500504x518S2e000520x485 % candy
M536x507S38900464x493 % ?
\vfil

\end{center}
\end{multicols}

\subsection{Story 7}

\begin{multicols}{5}
\begin{center}
M513x542S15402488x458S21100493x478S22c04490x500 % hungry
M518x518S10043488x483S20500482x507 % me
M536x504S38800464x496 % .

M534x543S14c30507x457S14c38469x458S15030508x512S15038467x511S26524493x493 % want
M528x537S33b00482x483S18507502x510S20500518x509 % eat
M536x504S38800464x496 % .

M535x522S22f14466x501S22f04510x501S2fb04494x516S18215468x479S1821d508x479 % now
M535x522S22f14466x501S22f04510x501S2fb04494x516S18215468x479S1821d508x479 % now
M518x518S10043488x483S20500482x507 % me
M513x531S1c501488x506S20e00491x489S22a00490x470 % feel
M525x517S20350510x483S20350476x483S22a24494x502 % can
M528x537S33b00482x483S18507502x510S20500518x509 % eat
M550x522S1f720530x506S11820451x478S2470a482x496 % pizza
M542x585S1e210520x532S1e218455x531S2ff00482x483S22a24492x563S2f900493x580 % CL: medium size circle pizza
M520x551S2e700504x483S15d09481x520S20500505x513S15d01493x528S15a20508x450 % all
M511x515S20600489x504S1f710491x485 % myself
M536x504S38800464x496 % .

M540x518S2ff00482x483S1f710520x494S20e00519x478S26500517x460 % tomorrow
M518x518S10043488x483S20500482x507 % me
M525x526S10018476x477S10018497x496S2882a503x475 % go
M542x523S10058459x493S14c10482x477S20500476x501S2a502509x490 % movie
M536x504S38800464x496 % .

M528x537S33b00482x483S18507502x510S20500518x509 % eat
M537x504S38700463x496 % ,
M528x537S33b00482x483S18507502x510S20500518x509 % eat
M537x504S38700463x496 % ,
M528x537S33b00482x483S18507502x510S20500518x509 % eat
M536x518S2b800511x454S15a10524x487S34700482x483 % will
M518x518S10043488x483S20500482x507 % me
M537x505S38730463x496 % !

M524x546S1eb07504x465S1eb0f476x476S2eb48480x508S21d00478x465S2eb00507x499S21d00500x454S2fd04497x540 % popcorn
M537x504S38700463x496 % ,
M560x513S17650482x487S17654507x487S20350545x488S20350440x487S22a06526x489S22a12460x488S2fb04495x507 % hotdog
M537x504S38700463x496 % ,
M541x542S2ff00482x483S10011510x512S20500504x518S2e000520x485 % candy
M537x504S38700463x496 % ,
M518x567S33b00482x483S16d10492x520S2d201490x544 % drink
M537x504S38700463x496 % ,
M536x518S2b800511x454S15a10524x487S34700482x483 % will
M518x570S2ff00482x483S14712488x519S22b00493x540 % full (under chin)
M518x518S10043488x483S20500482x507 % me
M536x504S38800464x496 % .

M513x514S15a01490x486S20500487x503 % my
M540x543S1c507499x518S20600518x508S2ff00482x483 % favorite
M540x537S33b00482x483S18507502x510S20600518x509 % food
M536x507S38900464x493 % ?

M523x537S15a3a490x494S20500482x481S20500513x481S2df08491x464S15051477x509 % cookie
M537x504S38700463x496 % ,
M515x511S21802485x493S16d40497x490 % milk
M549x574S16d18486x554S18511522x510S20700509x554S33b00482x483S24000494x519 % CL: dip in cup and eat
M536x504S38800464x496 % .

\end{center}
\end{multicols}

\end{document}

