\documentclass{article}

\usepackage{fontspec}
\usepackage{fullpage}
\usepackage{multicol}
\usepackage{multirow}
\usepackage{tikz}

\begin{document}

\newfontfamily\swfill{SuttonSignWritingFill.ttf}
\newfontfamily\swline{SuttonSignWritingLine.ttf}
\newcommand{\bul}{\hfil$\bullet$&}
\renewenvironment{glossary}{\begin{multicols}{5}\begin{center}}{\end{center}\end{multicols}}
\setcounter{secnumdepth}{0}
\setlength{\columnseprule}{1pt}

\section{Supplement For Lesson 12}

\subsection{Objectives}

\begin{tabular}{p{1cm}p{14cm}}
\bul I have completed the objectives for this lesson.\\
\bul I know which base symbols are in Symbol Group eyes.\\
\bul I know which base symbols are in Symbol Group middle.\\
\bul I am able to write and sign the Index Middle Baby on Circle handshape.\\
\bul I am able to write and sign the Index Ring Baby on Circle handshape.\\
\bul I am able to write and sign the Index Ring Baby on Angle handshape.\\
\bul I am able to write and sign the Middle Hinge handshape.\\
\bul I am able to draw tongue.\\
\bul I am able to recognize the vocabulary for this lesson.\\
\bul I am able to read the practice sentences for this lesson.\\
\bul I am able to read the practice story for this lesson.\\
\end{tabular}

\subsection{Symbol Group Eyes}

The twenty-third Symbol Group we informally call eyes.
Its official name is ``Brow Eyes Eyegaze'', but you can now list off the symbol groups as ``one, two, three, four, five, six, seven, eight, nine, thumb, contact, fingers, wall, diagonal, floor, curve wall, hit wall, hit floor, curve floor, circles, timing, head, eyes.''

\begin{center}
\begin{tabular}{rcrc}
\textbf{Base Symbol}&\textbf{Example}&\textbf{Base Symbol}&\textbf{Example}\\
Eyebrows Straight Up            &B518x518S30a00482x483&Eyebrows Straight Neutral      &B518x518S30b00482x483\\
Eyebrows Straight Down          &B518x518S30c00482x483&Dreamy Eyebrows Neutral Down   &B518x518S30d00482x483\\
Dreamy Eyebrows Down Neutral    &B518x518S30e00482x483&Dreamy Eyebrows Up Neutral     &B518x518S30f00482x483\\
Dreamy Eyebrows Neutral-Up      &B518x518S31000482x483&Forehead Neutral               &B518x518S31100482x483\\
Forehead Contact                &B518x518S31200482x483&Forehead Wrinkled              &B518x518S31300482x483\\
Eyes Open                       &B518x518S31400482x483&Eyes Squeezed                  &B518x518S31500482x483\\
Eyes Closed                     &B518x518S31600482x483&Eye Blink Single               &B518x518S31700482x483\\
Eye Blinks Multiple             &B518x518S31800482x483&Eyes Half Open                 &B518x518S31900482x483\\
Eyes Wide Open                  &B518x518S31a00482x483&Eyes Half Closed               &B518x518S31b00482x483\\
Eyes Widening Movement          &B518x518S31c00482x483&Eye Wink (Squeezed Eye Blink)  &B518x518S31d00482x483\\
Eyelashes Up                    &B518x518S31e00482x483&Eyelashes Down                 &B518x518S31f00482x483\\
Eyelashes Fluttering            &B518x518S32000482x483&Eyegaze Straight Wall Plane    &B518x518S32100482x483\\
Eyegaze Straight Wall Double    &B518x518S32200482x483&Eyegaze Straight Wall Alternate&B518x518S32300482x483\\
Eyegaze Straight Floor Plane    &B518x518S32400482x483&Eyegaze Straight Floor Double  &B518x518S32500482x483\\
Eyegaze Straight Floor Alternate&B518x518S32600482x483&Eyegaze Curved Wall Plane      &B518x518S32700482x483\\
Eyegaze Curved Floor Plane      &B518x518S32800482x483&Eyegaze Circles Wall Plane     &B518x518S32900482x483\\
\end{tabular}
\end{center}

\subsection{Symbol Group Middle}

The twenty-fourth Symbol Group we informally call middle.
It's official name is ``Cheeks Ears Nose Breath'', but you can now list off the symbol groups as ``one, two, three, four, five, six, seven, eight, nine, thumb, contact, fingers, wall, diagonal, floor, curve wall, hit wall, hit floor, curve floor, circles, timing, head, eyes, middle.''

\begin{center}
\begin{tabular}{rcrc}
\textbf{Base Symbol}&\textbf{Example}&\textbf{Base Symbol}&\textbf{Example}\\
Cheeks Puffed           &B518x518S32a00482x483&Cheeks Neutral          &B518x518S32b00482x483\\
Cheeks Sucked           &B518x518S32c00482x483&Tense Cheeks High       &B518x518S32d00482x483\\
Tense Cheeks Middle     &B518x518S32e00482x483&Tense Cheeks Low        &B518x518S32f00482x483\\
Ears                    &B524x518S33000476x483&Nose Neutral            &B518x518S33100482x483\\
Nose Contact            &B518x518S33200482x483&Nose Wrinkles           &B518x518S33300482x483\\
Nose Wiggles            &B518x518S33400482x483&Air Blowing Out         &B524x518S33500476x483\\
Air Sucking In          &B524x518S33600476x483&Air Blow Small Rotations&B509x507S33700491x494\\
Air Suck Small Rotations&B509x507S33800491x494&Breath Exhale           &B506x510S33900495x491\\
Breath Inhale           &B506x510S33a00495x491\\
\end{tabular}
\end{center}

\subsection{The Index Middle Baby on Circle Handshape}

The index middle baby on circle handshape starts as the spread fingers handshape and forms a circle with the thumb and ring finger.

\begin{center}
\begin{tabular}{r*{6}{c}}
\begin{tabular}{p{1cm}p{14cm}}
&\textbf{Fill 1}&\textbf{Fill 2}&\textbf{Fill 3}&\textbf{Fill 4}&\textbf{Fill 5}&\textbf{Fill 6}\\
\textbf{Right}&
\bul I am able to write and sign the Index Middle Baby on Circle handshape.\\
\bul I am able to write and sign the Index Ring Baby on Circle handshape.\\
\bul I am able to write and sign the Index Ring Baby on Angle handshape.\\
\bul I am able to write and sign the Middle Hinge handshape.\\
\bul I am able to draw tongue.\\
\bul I am able to recognize the vocabulary for this lesson.\\
\bul I am able to read the practice sentences for this lesson.\\
\bul I am able to read the practice story for this lesson.\\
\end{tabular}
\subsection{The Index Ring Baby on Circle Handshape}

The index ring baby on circle handshape starts as the spread fingers handshape and forms a circle with the thumb and middle finger.

\subsection{The Index Ring Baby on Angle Handshape}

The index ring baby on angle handshape starts as the spread fingers handshape, bends the middle finger to point the same direction as the palm is facing and brings the thumb up to touch the middle finger.

\subsection{The Middle Hinge Handshape}
\begin{center}\textbf{\Huge This is where we are!}\end{center}\end{document}

The middle hinge handshape starts as the spread fingers handshape, bends the middle finger to point the same direction as the palm is facing.

